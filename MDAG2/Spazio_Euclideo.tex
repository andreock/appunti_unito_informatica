\documentclass[a4paper, 10pt]{article}
\usepackage{geometry}

\usepackage[table]{xcolor}
\setlength{\parindent}{10pt}
\usepackage{amsfonts}
\usepackage{amsmath,amssymb}
% Choose a conveniently small page size
\usepackage{setspace}
% Using \doublespacing in the preamble 
% changes text to double line spacing
\onehalfspacing
\usepackage{listings}
\usepackage{color}

\definecolor{dkgreen}{rgb}{0,0.6,0}
\definecolor{gray}{rgb}{0.5,0.5,0.5}
\definecolor{mauve}{rgb}{0.58,0,0.82}

\lstset{frame=tb,
	language=C,
	aboveskip=3mm,
	belowskip=3mm,
	showstringspaces=false,
	columns=flexible,
	basicstyle={\small\ttfamily},
	numbers=none,
	numberstyle=\tiny\color{gray},
	keywordstyle=\color{blue},
	commentstyle=\color{dkgreen},
	stringstyle=\color{mauve},
	breaklines=true,
	breakatwhitespace=true,
	tabsize=3
}


\title{Spazio euclideo}
\author{Andrea Canale}

\begin{document}
	\maketitle
	\tableofcontents
	
\section{Isometrie}

Un isometria è un isomorfismo $T: V \rightarrow W$(con $g \text{ e } g^{'}$ prodotti scalari) che soddisfa la seguente proprietà:

$$g\left(v_1,v_2\right)=g'\left(T\left(v_1\right),T\left(v_2\right)\right)$$

Un isometria mantiene la lunghezza dei vettori invariata.

Ci sono altri 3 modi per verificarla:

\begin{itemize}
	\item Esiste una base $B$ tale che vale la proprietà dell'isometria($\forall v_i, v_j \in B$)
	\item $\forall v \in V$, $||v|| = ||T(v)||$
	\item $\forall v, w \in W$, $d(v,w) = d(T(v), T(w))$
\end{itemize}

Inoltre, $T$ è isometria se vale:
$$[g]_B = [T]^B_B \cdot ^t[g^{'}]_B \cdot [T]^B_B$$

Questo ci torna utile per l'endomorfismo $L_a: \mathbb{R}^n \rightarrow \mathbb{R}^n$ che è isometria se:

$$^t A \cdot A = I_n$$

\section{Isometrie su $\mathbb{R}^2$}

\subsection{Rotazioni}

Una rotazione di angolo $\theta$ è una mappa $L_a: \mathbb{R}^2 \rightarrow \mathbb{R}^2$ definita dalla matrice di rotazione $A=\left(\begin{matrix}cos\left(\theta\right)&-sin\left(\theta\right)\\sin\left(\theta\right)&cos\left(\theta\right)\\\end{matrix}\right)
$

\subsection{Riflessioni}

Una riflessione di angolo $\theta$ rispetto ad una retta $r$ è una mappa $L_a: \mathbb{R}^2 \rightarrow \mathbb{R}^2$ definita dalla matrice di rotazione $A=\left(\begin{matrix}cos\left(\theta\right)&sin\left(\theta\right)\\sin\left(\theta\right)&-cos\left(\theta\right)\\\end{matrix}\right)
$
\\\\
\textbf{Queste sono le uniche isometrie in $\mathbb{R}^2$}

\section{Matrici ortogonali}

Una matrice $M(n, \mathbb{K})$ è ortogonale se
$$^t A \cdot A = I_n$$ Ed equivalentemente: $^tA = A^{-1}$

Inoltre, se una matrice è ortogonale sappiamo che il determinante sarà $+1$ se la matrice descrive una rotazione o $-1$ se descrive una riflessione.

\textbf{Tuttavia non tutte le matrici con determinante $\pm 1$ sono ortogonali.}

\section{Isometrie su $\mathbb{R}^3$}

\subsection{Antirotazione}

Un antirotazione è la composizione di una rotazione rispetto ad un asse $r$ e una riflessione rispetto al piano $r^\perp$
\\
\textbf{In $\mathbb{R}^3$ ogni isometria è una rotazione o un'antirotazione}

\section{Prodotto vettoriale}

In $\mathbb{R}^3$ possiamo definire un operazione che funziona solo su questo spazio:

$$v\ x\ w=\left(\begin{matrix}v_2w_3-v_3w_2\\v_3w_1-v_1w_3\\v_1w_2-v_2w_1\\\end{matrix}\right)$$
Questa operazione può essere ricordata con questa formula: $v\ x\ w=det\left(\begin{matrix}e_1&e_2&e_3\\v_1&v_2&v_3\\w_1&w_2&w_3\\\end{matrix}\right)
$ sviluppando sulla prima riga e prendiamo i coefficienti dei rappresentanti canonici come componenti del vettore.

\subsection{Proprietà del prodotto vettoriale}

\begin{itemize}
	\item è ortogonale sia a $v$ che a $w$
	\item Se $v\ x\ w$ è nullo, allora $v$ e $w$ sono dipendenti
	\item Se $v$ e $w$ sono indipendenti, $v, w, v\ x\ w$ è una base positiva di $\mathbb{R}^3$
	\item $||v\ x\ w|| = ||v|| \cdot ||w|| - <v,w>$
	\item $||v\ x\ w|| = ||v|| \cdot ||w|| - sen(\theta)$
\end{itemize}

Le ultime due proprietà valgono anche per la norma al quadrato(mettendo al quadrato tutti i membri dell'uguaglianza)

\section{Basi positive}

Una base $\{v_1, v_2, v_3\}$ in $\mathbb{R}^3$ è positiva se:

$$det(v_1 | v_2 | v_3) > 0$$

\textbf{La positività di una base dipende dai vettori e dalla loro posizione.}

\section{Sottospazio affine}

Un sottospazio affine di uno spazio vettoriale $V$, dato un vettore $v_0$ è un sottospazio del tipo:
$$\{w \in W\ |\ v_0 + w\}$$

Due spazi $x+W$ e $y + V$ coincidono se e solo se $W=V$ e $x-y \in W$

Lo spazio $W$ di uno spazio affine $S$ scritto come $x+W$ è detto giacitura di $S$ ed è indicato come $giac(S)$
\subsection{Forma parametrica e cartesiana di spazi affini}

La forma cartesiana descrive uno spazio attraverso i punti che soddisfano un'equazione lineare omogenea:

è utile per determinare i punti di una funzione.

$$x=\{v\in\mathbb{R}^3|F(v)=0\}$$

La forma parametrica descrive uno spazio attraverso i vettori che lo generano:

$$x=F(y) \text{ Per qualche } F:y \rightarrow \mathbb{R}^3$$

Oppure:

$$\{\begin{pmatrix}
	0 \\
	1 \\
	2
\end{pmatrix}, t\begin{pmatrix}
1 \\
0 \\
1
\end{pmatrix}\}$$


è utile per determinare se un punto appartiene allo spazio.

Ad esempio con la bisettrice abbiamo:

\begin{itemize}
	\item In forma cartesiana: $\{(x,y) \in \mathbb{R}^2 | x-y = 0\}$
	\item In forma parametrica $\{(t,t) \in \mathbb{R}^2 | t \in \mathbb{R}\}$
\end{itemize}

Entrambi descrivono i punti con le stesse coordinate

\subsection{Intersezione tra spazi affini}

Per calcolare l'intersezione di due spazi affini $S \cap S^{'}$ dobbiamo sempre calcolare un sistema di equazioni indipendentemente dalla forma.
\end{document}