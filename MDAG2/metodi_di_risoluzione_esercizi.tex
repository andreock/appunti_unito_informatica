\documentclass[a4paper]{article}
\usepackage{graphicx}
\graphicspath{ {./esercizi_risoluzione/} }
\usepackage[table]{xcolor}
\setlength{\parindent}{10pt}
\usepackage{amsfonts}
\usepackage{amsmath,amssymb}
% Load the setspace package
\usepackage{setspace}
% Using \doublespacing in the preamble 
% changes text to double line spacing
\onehalfspacing
\usepackage{setspace} 
\title{Metodi di risoluzione esercizi}
\author{Andrea Canale}

\begin{document}
\maketitle
\tableofcontents

\section{Esercizi span}

Per trovare il sottospazio generato dobbiamo controllare che l'elemento sia definito nel sottospazio e  la dipendenza delle soluzioni proposte

Ad esempio dato questo esercizio: 

\includegraphics[scale=0.5]{span1}

Possiamo escludere gli elementi $b(x) $ e $ e(x) $ in quanto non rispettano condizione $p(1) = 0$

Poi troviamo che $c$ ed $f$ sono indipendenti(in questo caso si vede senza fare il sistema lineare)

Infine notiamo che $d(x) $ viene generato da $a(x) \cdot x$ quindi $d(x)$ va escluso

Concludiamo che la risposta corretta è la b: $Span(a,c,f)$

\subsection{Trovare combinazione lineare con polinomi}

Per trovare una combinazione lineare tra polinomi, come ad esempio: 

\includegraphics[scale=0.4]{comb_lin_polinomi}

Bisogna calcolare:

\begin{itemize}
	\item $\lambda_1(p_1(x))$
	\item $\lambda_2(p_2(x))$
	\item $\lambda_3(p_3(x))$
\end{itemize}

Otteniamo: $$ \lambda_1 x^2 + \lambda_1 x + \lambda_1 + \lambda_2 x^2 + \lambda_2 x - \lambda_2 + \lambda_3 x - 2\lambda_3 = (x+1)^2 $$

Ora raccogliamo $x^2, x, \text{termini noti} $:

$$(\lambda_1 + \lambda_2) x^2 + (\lambda_1 + \lambda_2 + \lambda_3)x + \lambda_1 - \lambda_2 - 2\lambda_3 = x^2 + 2x + 1$$

Adesso possiamo trasferire il tutto in un sistema lineare:

\begin{equation*}
	\left\{
	\begin{alignedat}{3}
		\lambda_1 & +{} & \lambda_2 & = 1 \\
		\lambda_1 & +{} & \lambda_2 & +{} & \lambda_3 & = 2 \\
		\lambda_1 & -{} & \lambda_2 & -{} & 2\lambda_3 & = 1
	\end{alignedat}
	\right.
\end{equation*}

Dove nella prima riga abbiamo isolato ed eguagliato $x^2$, nella seconda $x$ e nella terza i termini noti.

Ora possiamo risolvere il sistema lineare e trovare le soluzioni che in questo caso saranno:

\begin{itemize}
	\item $\lambda_1 = 2$
	\item $\lambda_2 = -1$
	\item $\lambda_3 = 1$
\end{itemize}

Che corrispondono alla risposta (d)
\section{Esercizi sottospazio}

Per trovare un sottospazio, dobbiamo controllare le sue 3 condizioni, ad esempio dato il seguente esercizio:

\includegraphics[scale=0.4]{sottospazio1}

Bisogna controllare le 3 condizioni per tutti i sottospazi dati
\end{document}