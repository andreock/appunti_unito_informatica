\documentclass[a4paper, 10pt]{article}
\usepackage{geometry}

\usepackage[table]{xcolor}
\setlength{\parindent}{10pt}
\usepackage{amsfonts}
\usepackage{amsmath,amssymb}
% Choose a conveniently small page size
\usepackage{setspace}
% Using \doublespacing in the preamble 
% changes text to double line spacing
\onehalfspacing
\usepackage{listings}
\usepackage{color}
\usepackage[table]{xcolor}
\setlength{\parindent}{10pt}
\usepackage{amsfonts}
\usepackage{amsmath,amssymb}
\usepackage{soul}
\usepackage{xcolor}
\usepackage[overload]{empheq} % for Solution 2

\usepackage{setspace}
% Using \doublespacing in the preamble
% changes text to double line spacing
\doublespacing
\usepackage{setspace}
\usepackage{listings}
\usepackage{color}

\title{Autovalori e Autovettori}
\author{Andrea Canale}

\begin{document}
	\maketitle
	\tableofcontents

\section{Autovettori}

Dato $T:V \rightarrow V$ un endomorfismo di uno spazio vettoriale V definito in un campo $\mathbb{K}$, un autovettore di T è un vettore $v \neq 0$ tale che $T(v)=\lambda v$ per qualche $\lambda \in \mathbb{K}$ che chiameremo autovalore associato a v.

Notiamo che se $\lambda = 0$, $v \in ker(V)$

Inoltre, ogni multiplo di un autovettore è a sua volta un autovettore con un autovalore diverso, ad esempio:

$T: \mathbb{R}^3 \rightarrow \mathbb{R}^3$, definita come $T\begin{pmatrix}
	x \\
	y \\
	z 
\end{pmatrix} = \begin{pmatrix}
1 & 1 & 1 \\
2 & 1 & 1 \\
3 & 0 & 2 
\end{pmatrix}\begin{pmatrix}
x \\
y \\
z 
\end{pmatrix} $

Abbiamo come autovettori

\begin{itemize}
	\item $v=\begin{pmatrix}
		0 \\
		1 \\
		1 
	\end{pmatrix}$, $T(v) = \begin{pmatrix}
		0 \\
		2 \\
		2 
	\end{pmatrix} = 2v$
	\item $w=\begin{pmatrix}
		0 \\
		3 \\
		3 
	\end{pmatrix}$, $T(w) = \begin{pmatrix}
		0 \\
		6 \\
		6 
	\end{pmatrix} = 2w = 6v$
\end{itemize} 

\section{Matrici diagonalizzabili}

\subsection{Diagonalizzazione per endomorfismi}
Un endomorfismo $T: V \rightarrow V$ è diagonalizzabile se V ha una base B composta dai suoi autovettori e se $[T]^B_B$ è composta da autovalori di T sulla diagonale.

Esempio:

$T: \mathbb{R}^2 \rightarrow \mathbb{R}^2$ definita come $T\left(\left(\begin{matrix}x\\y\\\end{matrix}\right)\right)=\left(\begin{matrix}3&4\\0&2\\\end{matrix}\right)\left(\begin{matrix}x\\y\\\end{matrix}\right)$

Troviamo che i suoi autovettori sono

\begin{itemize}
	\item $v_1=\begin{pmatrix}
		1 \\
		0
	\end{pmatrix}, T(v_1) = \begin{pmatrix}
	3 \\
	0
	\end{pmatrix} = 3v_1$
	\item  $v_2=\begin{pmatrix}
		-4 \\
		1
	\end{pmatrix}, T(v_2) = \begin{pmatrix}
	-8 \\
	2
	\end{pmatrix} = 2v_2$
\end{itemize}

Gli autovalori sono 3 e 2. Notiamo ora che $v_1, v_2$ formano una base di $\mathbb{R}^2$.

Calcoliamo la matrice associata a T(la matrice delle coordinate dei risultati di T(v1) e T(v2)): $[T]_B^B=\begin{pmatrix}
	3 & 0 \\
	0 & 2
\end{pmatrix}$

Cioè: $3v_1 + 0v_2 = T(v_1)$ e $0v_1 + 2v_2 = T(v_2)$

Concludiamo che l'endomorfismo è diagonalizzabile.

\subsection{Diagonalizzazione in generale}

La teoria che abbiamo visto prima vale solo per gli endomorfismi. Vediamo ora come diagonalizzare una matrice qualsiasi.

Una matrice $A \in M(n, \mathbb{K})$ è diagonalizzabile, se è simile ad una matrice D che è diagonale: $$D=M^{-1} \cdot A \cdot M$$

Ciò vale anche per gli endomorfismi, un endomorfismo è diagonalizzabile se la sua matrice associata è diagonalizzabile.

\section{Matrici diagonali}

Il motivo per cui scegliamo di usare matrici diagonali è che ci semplifica i calcoli:

\begin{itemize}
	\item Il calcolo del determinante è il prodotto degli elementi sulla diagonale
	\item Il prodotto fra una matrice e un vettore è $\begin{pmatrix}
		\lambda_1 & 0 & \dots & 0 \\
		0 & \lambda_2 &  \dots & 0 \\
		\vdots & \vdots & \ddots & \vdots \\
		0 & 0 & \dots & \lambda_n
	\end{pmatrix} \begin{pmatrix}
	x_1 \\
	x_2 \\
	\vdots \\
	x_n
	\end{pmatrix} = \begin{pmatrix}
	\lambda_1 x_1 \\
	\lambda_2 x_2 \\
	\vdots \\
	\lambda_n x_n
	\end{pmatrix}$
	\item Possiamo facilmente calcolare potenze molto grandi: $(M^{-1} \cdot A \cdot M)^{100}$
\end{itemize}

\section{Polinomio caratteristico}

Per trovare autovalori usiamo il polinomio caratteristico, definito come:

$$P_a(\lambda) = det(A-\lambda I_n)=det(\begin{pmatrix}
	a_{11} - \lambda & \dots & a_{1n} \\
	a_{21} & \ddots  & \vdots \\
	a_{n1} & \dots & a_{nn} - \lambda
\end{pmatrix})$$

Ossia il determinante della matrice di cui vogliamo trovare gli autovalori togliendo $\lambda$ sulla diagonale.

Infatti $I_n$ è l'identità di grandezza $n$ dove $n$ è la dimensione della $A \in M(n, \mathbb{K})$.

Questo calcolo restituisce un polinomio di grano n con incognite $\lambda$. 

Notiamo che se il polinomio viene di un grado minore di $n$, c'è sicuramente un errore nei calcoli.

\subsection{Polinomi caratteristici di matrici simili}

Se A e B sono simili, allora $P_a(\lambda) = P_b(\lambda)$

\subsection{Polinomio caratteristico di un endomorfismo}

Dato un endomorfismo $T:V \rightarrow V$, il polinomio caratteristico di T è $P_a(\lambda)$ dove $A=[T]^B_C$.

\section{Autovettori con autovalori distinti}

Se $v_1, ..., v_k$ sono autovettori con autovalori $\lambda_1, ..., \lambda_k$ distinti, allora $v_1, ..., v_k$ sono linearmente indipendenti.

Inoltre, se $P_a(\lambda)$ ha n radici distinte, allora A è diagonalizzabile.

\section{Autospazio}

Sia T un endomorfismo. Per ogni autovalore $\lambda$ definiamo l'autospazio:

$$V_{\lambda}=\{v \in V | T(v) = \lambda v\} = ker(T-\lambda id)$$

In altre parole l'autospazio è l'insieme di tutti gli autovettori che hanno autovalore $\lambda$ più l'origine $0_v$ che non sarebbe un autovettore(perchè gli autovettori sono diversi da 0).

Gli autospazi dei corrispettivi autovalori sono sempre in somma diretta.

%Da ciò ne ricaviamo che l'endomorfismo $T: V \rightarrow V$ è diagonalizzabile se $V=V_{\lambda_1} \oplus ... \oplus V_{\lambda_k}$.

\section{Molteplicità algebrica}

Sia $T: V \rightarrow V$ un endomorfismo e $\lambda$ un autovalore per T. La molteplicità algebrica, definita come $m_a(\lambda)$ è la molteplicità di $\lambda$ come radice del polinomio caratteristico.

In altre parole, corrisponde al numero di soluzioni trovare per ogni radice distinta di $p_a(\lambda)$.

\section{Molteplicità geometrica}

La molteplicità geometrica $m_g(\lambda)$ è la dimensione dell'autospazio associato a $\lambda$. Dato $T: V \rightarrow V$, possiamo calcolare $m_g(\lambda)$ attraverso il teorema della dimensione: $$m_g(\lambda) = dim(ker(A-\lambda I_n)) = dim(V) - rk(A- \lambda I_n)$$

Notiamo che per ogni autovalore $\lambda_i$ di un endomorfismo $T$, vale:

$$1 \leq m_g(\lambda_i) \leq m_a(\lambda_i)$$

Questo ci può tornare utile per controllare che i calcoli siano corretti e perchè se $m_a(\lambda) = 1$, allora sicuramente $m_g(\lambda) = 1$

\section{Teorema della diagonalizzabilità}

Sia V uno spazio vettoriale su $\mathbb{K}$ di dimensione n e $T: V \rightarrow V$ un endomorfismo, esso è diagonalizzabile se valgono due proposizione:

\begin{itemize}
	\item $p_t(\lambda)$ ha n radici distinte contate con molteplicità
	\item $m_g(\lambda) = m_a(\lambda)$ per ogni autovalore di T
\end{itemize}

\section{Sommario diagonalizzabilità}

Un endomorfismo è diagonalizzabile se:

\begin{itemize}
	\item Vale il teorema della diagonalizzabilità. Oppure
	\item Se $P_a(\lambda)$ ha n radici distinte, allora A è diagonalizzabile. Oppure
	\item $[T]^B_B$ è composta da autovalori di $T$ sulla diagonale  
\end{itemize}

\hl{In ogni caso, negli esercizi bisogna sempre dimostrare il teorema della diagonalizzabilità.}

Una matrice quadrata $M$ è diagonalizzabile se:

\begin{itemize}
	\item Esiste una matrice D tale che $D=M^{-1} \cdot A \cdot M$
\end{itemize}
\end{document}