\documentclass[a4paper, 10pt]{article}
\usepackage{geometry}

\usepackage[table]{xcolor}
\setlength{\parindent}{10pt}
\usepackage{amsfonts}
\usepackage{amsmath,amssymb}
% Choose a conveniently small page size
\usepackage{setspace}
% Using \doublespacing in the preamble 
% changes text to double line spacing
\onehalfspacing
\usepackage{listings}
\usepackage{color}
\title{Combinazioni lineari}
\author{Andrea Canale}

\begin{document}
	
\maketitle
\tableofcontents
\section{Combinazioni lineari}

Dato una spazio vettoriale $V$ e siano $v_1, ..., v_n$ vettori in $V$, una combinazione lineare di questi vettori è un vettore di forma:

$$v\ =\ \lambda_1v_1+\lambda_2v_2\ +...+\lambda_nv_n$$

Dove $\lambda_1, ..., \lambda_n \in \mathbb{K}$

Esempio

$$v\ =\ \lambda_1\left(\begin{matrix}1\\0\\0\\\end{matrix}\right)\ +\ \lambda_2\left(\begin{matrix}0\\1\\0\\\end{matrix}\right)\ =\ \left(\begin{matrix}\lambda_1\\0\\0\\\end{matrix}\right)\ +\ \left(\begin{matrix}0\\\lambda_2\\0\\\end{matrix}\right)\ =\left(\begin{matrix}\lambda_1\\\lambda_2\\0\\\end{matrix}\right)$$

\section{Sottospazio generato}

Dato uno spazio vettoriale $V$ in $\mathbb{K}$ e siano $v_1, ..., v_n$ vettori in $V$.

Il sottospazio generato da questi vettori è un sottoinsieme di $V$ formato da tutte le combinazioni lineari di questi vettori. Viene indicato come $$Span\left(v_1,\ ...,\ v_k\right)\ =\ \left\{\lambda_1v_1,\ ...,\ \lambda_kv_k\ |\lambda_1,\ ...,\ \lambda_k\ \in\mathbb{K}\right\}$$

\section{Indipendenza lineare}

Dato uno spazio vettoriale $V$ su $\mathbb{K}$ e siano $v_1, ..., v_n$ vettori in questo spazio.

I vettori si dicono \textbf{linearmente dipendenti} se esistono $\lambda_1, ..., \lambda_n$ con almeno un $\lambda \neq 0$ se
$$\lambda_1v_1,\ ...,\ \lambda_kv_k\ =\ 0$$

Questo vuol dire che i vettori si possono scrivere come combinazione lineare tra loro.

In caso contrario i vettori si dicono \textbf{linearmente indipendenti}

Per capire se i vettori sono linearmente indipendenti dobbiamo risolvere un sistema per trovare i coefficienti.

Questo si può fare facilmente utilizzando il teorema di Rouchè-Capelli:

\begin{itemize}
	\item Se abbiamo $\infty$ soluzioni, i vettori dipendenti
	\item Se abbiamo 1 soluzione, i vettori sono indipendenti perchè l'unica soluzione che esiste è sicuramente 0 perchè esiste sempre
\end{itemize}

Esempio

$$w_1\ =\ \left(\begin{matrix}1\\1\\2\\\end{matrix}\right),\ w_2\ =\ \left(\begin{matrix}1\\2\\1\\\end{matrix}\right),\ w_3\ \left(\begin{matrix}3\\2\\1\\\end{matrix}\right)=$$

$$ \lambda_1\left(\begin{matrix}1\\1\\2\\\end{matrix}\right)\ +\ \lambda_2\left(\begin{matrix}1\\2\\1\\\end{matrix}\right)\ +\ \lambda_3\left(\begin{matrix}3\\2\\1\\\end{matrix}\right)\ =\ \left(\begin{matrix}0\\0\\0\\\end{matrix}\right)\ = $$

$$ \left(\begin{matrix}\lambda_1\ +\ \lambda_2\ +\ 3\lambda_3\\\lambda_1\ +\ 2\lambda_2\ +\ \lambda_3\\2\lambda_1\ +\ \lambda_2\ +\ \lambda_3\\\end{matrix}\right)\ =\ \left(\begin{matrix}0\\0\\0\\\end{matrix}\right)\ = $$

E possiamo formare il seguente sistema in forma matriciale:

$$ \begin{pmatrix}
	1 & 1 & 3 & 0 \\
	1 & 2 & 1 & 0 \\
	2 & 1 & 1 & 0 
\end{pmatrix}  $$

% TODO: Calcolare il determinante

\section{Basi e dimensioni}

Data una sequenza di vettori $v_1, ..., v_n$ in V, essi formano una base se vengono soddisfatte due condizioni:

\begin{itemize}
	\item Questi vettori sono linearmente indipendenti
	\item $Span(v_1, ..., v_n) = V$ Cioè che il rango della matrice del sistema omogeneo associato con i vettori dello Span sia uguale alla dimensione di V
\end{itemize}

La dimensione di V, definita come $dimV$ è il numero di elementi nella base di V.

Due basi di uno stesso spazio hanno la stessa dimensione.
\subsection{Base canonica}

Dati vettori canonici(vettori con un unico 1 e nelle altre posizione 0), formati da $$e_1\ =\ \left(\begin{matrix}1\\\vdots\\0\\\end{matrix}\right),\ e_2\ =\ \left(\begin{matrix}0\\\vdots\\0\\\end{matrix}\right),\ e_2\ =\ \left(\begin{matrix}0\\\vdots\\1\\\end{matrix}\right) $$

Questi vettori canonici formano una base nello spazio $\mathbb{K}^n$ chiamata base canonica.

Infatti i vettori canonici formano lo spazio $\mathbb{K}^n$ 
\end{document}