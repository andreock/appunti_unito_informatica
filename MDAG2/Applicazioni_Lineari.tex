\documentclass[a4paper, 10pt]{article}
\usepackage{geometry}

\usepackage[table]{xcolor}
\setlength{\parindent}{10pt}
\usepackage{amsfonts}
\usepackage{amsmath,amssymb}
% Choose a conveniently small page size
\usepackage{setspace}
% Using \doublespacing in the preamble 
% changes text to double line spacing
\onehalfspacing
\usepackage{listings}
\usepackage{color}

\setlength{\parindent}{10pt}
\usepackage{amsfonts}
\usepackage{amsmath,amssymb}
\usepackage{soul}
\usepackage{xcolor}
\usepackage[overload]{empheq} % for Solution 2
% Load the setspace package
\usepackage{setspace}
% Using \doublespacing in the preamble
% changes text to double line spacing
\doublespacing
\usepackage{setspace}
\usepackage{listings}
\usepackage{color}

\definecolor{dkgreen}{rgb}{0,0.6,0}
\definecolor{gray}{rgb}{0.5,0.5,0.5}
\definecolor{mauve}{rgb}{0.58,0,0.82}

\lstset{frame=tb,
	language=C,
	aboveskip=3mm,
	belowskip=3mm,
	showstringspaces=false,
	columns=flexible,
	basicstyle={\small\ttfamily},
	numbers=none,
	numberstyle=\tiny\color{gray},
	keywordstyle=\color{blue},
	commentstyle=\color{dkgreen},
	stringstyle=\color{mauve},
	breaklines=true,
	breakatwhitespace=true,
	tabsize=3
}


\title{Applicazioni lineari}
\author{Andrea Canale}

\begin{document}
	\maketitle
	\tableofcontents
	
\section{Applicazioni lineari}

Dati due spazi vettoriali $V$ e $W$ sullo stesso campo $\mathbb{K}$, un'applicazione lineare è una funzione $f: V \rightarrow W$ tale che valgono i seguenti assiomi:

\begin{itemize}
	\item $f(0) = 0$
	\item $f(v+w) = f(v) + f(w) \ \forall v,w \in V$
	\item $f(\lambda v) = \lambda f(v) \ \forall \lambda \in \mathbb{K} \ \forall v \in V$
\end{itemize} 

Osserviamo che se $v=\lambda_1 v_1 + ... + \lambda_n v_n$, osserviamo che:

$f(v)=f(\lambda_1 v_1) + ... + f(\lambda_n v_n) = \lambda_1 f(v_1) + ... + \lambda_n f(v_n)$

E quindi possiamo unire le proprietà insieme.

\subsection{Funzioni note non lineari}

\begin{itemize}
	\item Funzioni che hanno un grado maggiore di 1, ad esempio: $f\left(\begin{matrix}a\\b\\\end{matrix}\right)=\left(\begin{matrix}2a^2+b\\a-2b\\\end{matrix}\right)
	$
	\item Funzioni che hanno termini noti, ad esempio: $f\left(\begin{matrix}a\\b\\\end{matrix}\right)=\left(\begin{matrix}a+2\\a-b-1\\\end{matrix}\right)$
\end{itemize}

\subsection{Funzioni note lineari}

\begin{itemize}
	\item La funzione nulla(composta da soli zeri)
	\item La funzione identità(1 sulla diagonale)
\end{itemize}

\section{Nucleo e immagine}

\subsection{Nucleo}
Il nucleo di una funzione $f: V \rightarrow W$ è definito come: 

$$ker(f) = \{v \in V | f(v) = 0_w\}$$

Sappiamo inoltre che il nucleo è sempre sottospazio di V perchè contiene l'origine 0 e i vettori $v \in V$

La funzione $f$ è iniettiva se e solo se $ker(f) = \{0\}$

\hl{Per trovare il nucleo bisogna risolvere il sistema f(v) = 0}

Esempio:

$$ f: \mathbb{R}^2 \rightarrow \mathbb{R}^3 \text{ definita come: } f(\begin{pmatrix}
	x \\
	y 
\end{pmatrix}) = \begin{pmatrix}
2x+y \\
x-y \\
3x+2y 
\end{pmatrix}  $$

Dobbiamo risolvere il sistema $\begin{pmatrix}
	2 & 1 & 0 \\
	1 & -1 & 0 \\
	3 & 2 & 0 
\end{pmatrix}$ e, in questo caso, avrebbe 0 soluzioni cioè il nucleo è vuoto.

Inoltre la dimensione del nucleo può essere calcolata come: $n-rk(A)$ dove $n$ è il numero di incognite del sistema lineare.

\subsection{Immagine}
L'immagine di una funzione $f: V \rightarrow W$ è definita come:

$$ Im = \{w \in W | \exists v \in V \text{ tale che } f(v)=w\}$$

L'immagine è sempre sottospazio di W perchè contiene l'origine 0 e i vettori $w \in W$

Inoltre sappiamo che la funzione $f$ è suriettiva se e solo se $Im(f) = W$

L'immagine può essere trovata calcolando lo Span delle colonne della matrice associata all'applicazione lineare.

Esempio:

$$ f: \mathbb{R}^2 \rightarrow \mathbb{R}^3 \text{ definita come: } f(\begin{pmatrix}
	x \\
	y 
\end{pmatrix}) = \begin{pmatrix}
	2x+y \\
	x-y \\
	3x+2y 
\end{pmatrix}  $$

Rispetto alla base canonica. 

\hl{Dobbiamo risolvere controllare che le colonne della matrice associata(in questo caso alla base canonica) generino uno Span. In questo caso abbiamo $\mathbb{R}^2$ e quindi otteniamo la matrice} $$\begin{pmatrix}
	2 & 1 \\
	1 & -1 \\
	3 & 2 
\end{pmatrix}$$. Notiamo che le colonne sono indipendenti tra loro e quindi $Span=\{\begin{pmatrix}
2 \\
1 \\
3 
\end{pmatrix}, \begin{pmatrix}
1 \\
-1 \\
2 
\end{pmatrix}\}$

Un ulteriore prova di ciò può essere fatto trovando $dim(Im(f)) = rk(f) = 2$
\iffalse
\subsection{Generatori dell'immagine}

Se $v_1,...,v_n$ sono generatori di V, allora $f(v_1),...,f(v_n)$ sono generatori di $I_m(f)$

Da ciò ne deduciamo due cose:

\begin{itemize}
	\item $Im f = Span(f(v_1), ..., f(V_n))$
	\item $Im L_a = Span(A^1, ..., A^n)$ dove A è una matrice $m\ \text{x}\ n$
\end{itemize}

Concludiamo che per ogni matrice A vale:
$$rk(A) = dim(Span(A^1, ..., A^n)) = dimImL_a$$
\fi

\section{Matrice associata all'applicazione lineare}

Sia $f: V \rightarrow W$ un'applicazione lineare, $B=\{v_1, ..., v_n\}$ una base di V e $C=\{w_1, ..., w_n\}$ una base di W, possiamo rappresentare qualsiasi vettore di V come combinazione lineare di C. Questo ci permette di rappresentare una funzione lineare come matrice. è denotata come: $[f]_C^B$

%%Per trovare le coordinate dei vettori di B in C, usiamo un sistema lineare dove eguagliamo i vettori in C ai vettori in B e risolviamo il sistema fino a trovare i coefficienti che saranno le nostre coordinate.

\textbf{Vale la seguente proprietà: $[f(v)]_C = [f]_C^B \cdot [v]_B$}

\section{Teorema della dimensione}

Sia $f: V \rightarrow W$ una funzione lineare. Se V ha dimensione finita n, allora:
$$dimKerf+dimImf=n$$

Nel caso di un applicazione lineare $L_a: \mathbb{K}^n \rightarrow \mathbb{K}^n$, abbiamo che:
$$kerL_a=\{x \in \mathbb{K}^n | Ax = 0\} = S$$

Ciò è lo spazio delle soluzioni del sistema $Ax = 0$. Per calcolare la dimensione di questo spazio, possiamo usare il teorema di Rouchè-Capelli:
$$dimS = n - rk(A)$$

\section{Classificazioni}
\begin{itemize}
	\item Se $dimKer(f) \neq 0$, allora f è iniettiva
	\item Se $dimIm(f) \geq dimW$, allora f è suriettiva
	\item Se $dimW = dimV$, allora f è iniettiva
\end{itemize}

\section{Isomorfismi}

Un isomorfismo è un applicazione lineare biettiva. Due spazi vettoriali V, W sono isomorfi se esiste un isomorfismo $f: V \rightarrow W$. Inoltre dato che l'isomorfismo è biettivo, esisterà anche una funzione inversa.

Quindi due spazi vettoriali W,V sono isomorfi, se e solo se, $dimW = dimV$.

Da ciò deduciamo che tutti gli spazi vettoriali su $\mathbb{K}$ sono isomorfi rispetto a $\mathbb{K}^n$

\section{Matrice del cambiamento di base}

Sia V uno spazio vettoriale, B e C due basi di V. La matrice del cambiamento di base da B in C è definita come: $A=[id_v]_C^B$

Questa matrice contiene nella sua j-esima colonna, le coordinate di $v_j$ rispetto alla base C.

La matrice del cambiamento di base quindi codifica nella sue colonne le coordinate di ciascun elemento rispetto a B.

Per calcolare la matrice del cambiamento di base ci sono due tecniche:

\begin{itemize}
	\item Scrivere un sistema lineare dove cerchiamo eguagliamo i vettori di C a quelli di B e troviamo i coefficienti adatti a rendere vera l'uguaglianza
	\item Usare la base canonica.
\end{itemize}

Il metodo che utilizzeremo prevederà l'utilizzo della base canonica.

Esempio:

Dato l'endomorfismo $T=id: \mathbb{R}^2 \rightarrow \mathbb{R}^2$, scrivere la matrice associata rispetto alle base $B=\{\begin{pmatrix}
	2 \\
	4 
\end{pmatrix}, \begin{pmatrix}
4 \\
2 
\end{pmatrix}\}$ e $C=\{\begin{pmatrix}
2 \\
0 
\end{pmatrix}, \begin{pmatrix}
0 \\
-2 
\end{pmatrix}\}$

Data la base canonica $E=\{e_1, e_2\}$, i vettori in base canonica saranno: $$ [id]^B_E=\begin{pmatrix}
2 & 4 \\
4 & 2 
\end{pmatrix}  $$

e 

$$[id]^C_E=\begin{pmatrix}
	2 & 0 \\
	0 & -2 
\end{pmatrix}$$

\hl{Adesso la matrice del cambiamento di base si può ottenere facendo:} $$[id]^B_C=[id]^B_E \cdot [id]^E_C=[id]^B_E \cdot ([id]^C_E)^{-1} \cdot A$$

Dove A è la matrice associata all'applicazione lineare(in questo caso $I_2$ quindi la omettiamo)

Ed otteniamo $$\begin{pmatrix}
	2 & 4 \\
	4 & 2 
\end{pmatrix} \cdot \begin{pmatrix}
\frac{1}{2} & 0 \\
0 & -\frac{1}{2} 
\end{pmatrix} = \begin{pmatrix}
1 & -2 \\
2 & -1 
\end{pmatrix}$$

\section{Composizione di applicazioni lineari}

Se due funzioni f e g sono lineari, la loro composizione sarà lineare.

Inoltre vale la seguente proprietà: $L_a \cdot L_b = L_{ab}$

\section{Endomorfismi}

Sia V uno spazio vettoriale, un endomorfismo è un'applicazione lineare $f: V \rightarrow V$.

Da ciò ne ricaviamo che la matrice associata ad f con B una base di V è $[f]^B_B$

\section{Matrici simili}

Due matrici A, B sono simili se esiste una matrice invertibile tale che $A=M^{-1} \cdot B \cdot M$.

Se due matrici sono simili allora:

\begin{itemize}
	\item Il loro rango è uguale
	\item Il loro terminante è uguale
\end{itemize}

Inoltre se A è invertibile anche B è invertibile.
\end{document}