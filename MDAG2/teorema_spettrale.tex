\documentclass[a4paper, 10pt]{article}
\usepackage{geometry}

\usepackage[table]{xcolor}
\setlength{\parindent}{10pt}
\usepackage{amsfonts}
\usepackage{amsmath,amssymb}
% Choose a conveniently small page size
\usepackage{setspace}
% Using \doublespacing in the preamble 
% changes text to double line spacing
\onehalfspacing
\usepackage{listings}
\usepackage{color}

\definecolor{dkgreen}{rgb}{0,0.6,0}
\definecolor{gray}{rgb}{0.5,0.5,0.5}
\definecolor{mauve}{rgb}{0.58,0,0.82}

\lstset{frame=tb,
	language=C,
	aboveskip=3mm,
	belowskip=3mm,
	showstringspaces=false,
	columns=flexible,
	basicstyle={\small\ttfamily},
	numbers=none,
	numberstyle=\tiny\color{gray},
	keywordstyle=\color{blue},
	commentstyle=\color{dkgreen},
	stringstyle=\color{mauve},
	breaklines=true,
	breakatwhitespace=true,
	tabsize=3
}


\title{Teorema Spettrale}
\author{Andrea Canale}

\begin{document}
	\maketitle
	\tableofcontents
	
\section{Prodotti hermitiani}

Sia V uno spazio vettoriale complesso. Un prodotto hermitiano $V\ x\ V \rightarrow \mathbb{C}$ con le proprietà:

\begin{itemize}
	\item $\textlangle v + v^{'}, w \textrangle = \textlangle v, w \textrangle + \textlangle v^{'}, w \textrangle$
	\item $\textlangle \lambda v, w \textrangle = \lambda \textlangle v, w \textrangle$
	\item $\textlangle w, v \textrangle = \overline{ \textlangle v, w \textrangle}$
	\item $\textlangle v, v \textrangle = \overline{ \textlangle v, v \textrangle}$
	\item $\textlangle v, w + w^{'} \textrangle = \textlangle v, w \textrangle + \textlangle v, w^{'}\textrangle$
	\item $\textlangle v, 0 \textrangle = \textlangle 0, w \textrangle = 0$
\end{itemize}

Questo prodotto è sequilineare perchè è lineare sul primo fattore ma sul secondo fattore è lineare solo per somma.

\subsection{Prodotto hermitiano definito positivo}

Un prodotto hermitiano è definito positivo se $\textlangle v, v \textrangle \textgreater 0$ $\forall v \neq 0$

Inoltre possiamo definire la norma di un prodotto hermitiano definito positivo come: $$||v|| = \sqrt{\textlangle v, v \textrangle} = 0 \text{ se v $\neq$ 0}$$ 

\subsection{Prodotto hermitiano euclideo}

Il prodotto hermitiano euclideo è definito come: $$\textlangle x,y \textrangle = ^tx \cdot \overline{y}$$
\section{Matrici hermitiane}

Una matrice $H \in M(n, \mathbb{C})$ è hermitiana se: $$^tH = \overline{H}$$

\textbf{Notiamo che sulla diagonale di una matrice hermitiana abbiamo sempre numeri reali.}

Inoltre, se tutte le entrate sono reali, allora H è hermitiana se e solo se H è simmetrica.

\section{Matrice associata ad un prodotto hermitiano}

Sia V uno spazio vettoriale complesso e $B=\{v_1, ..., v_n\}$, una base di V. La matrice associata al prodotto hermitiano è data da: $$H_{ij} = \textlangle v_i, v_j \textrangle$$

Questa matrice associata a $g$ in $B$ è scritta come: $[g]_B$

\section{Endomorfismi autoaggiunti}

Un endomorfismo $T: V \rightarrow V$ è autoaggiunto se: $$\textlangle T(v), w \textrangle = \textlangle v, T(w) \textrangle$$

Data $A$ la matrice di $T$ associata a $B$, T è autoaggiunto, se e solo se, $A$ è hermitiana.
\\\\
\textbf{Corollario}

\begin{itemize}
	\item Sia $A \in M(n, \mathbb{C})$, l'endomorfismo $L_a: \mathbb{C}^n \rightarrow \mathbb{C}^n$ è autoaggiunto, se e solo se $A$ è hermitiana
	\item Sia $A \in M(n, \mathbb{C})$, l'endomorfismo $L_a: \mathbb{R}^n \rightarrow \mathbb{R}^n$ è autoaggiunto, se e solo se $A$ è simmetrica
\end{itemize}
Questo corollario vale solo per il prodotto hermitiano/scalare euclideo.

\section{Sottospazi invarianti}

Dato uno spazio $V$ con prodotto scalare/hermitiano definito positivo, il sottospazio $U \subset V$ è invariante se $$T(U) \subset U$$

Notiamo che se T è autoaggiunto e $U$ è un sottospazio invariante, allora anche $U^\perp$ è T-invariante

\section{Teorema spettrale}

Un endomorfismo $T: V \rightarrow V$ è autoaggiunto se e solo se V ha una base ortogonale formata da autovettori di T e se tutti questi autovettori sono reali.

Per le matrici reali questi fatti sono equivalenti:

\begin{itemize}
	\item A è simmetrica
	\item $L_a$ è autoaggiunto rispetto al prodotto scalare euclideo
	\item $L_a$ e $A$ ha una base ortogonale di autovettori(con autovalori reali)
	\item Esiste una matrice ortogonale $M$ tale che $M^{-1} \cdot A \cdot M = D$ che è diagonale
\end{itemize}

Questo ci torna molto utile per capire se una matrice ha una base ortogonale, infatti basta verificare che sia simmetrica perchè valgano tutte le altre proprietà del corollario.
\end{document}