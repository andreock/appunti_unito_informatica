\documentclass[a4paper, 10pt]{article}
\usepackage{geometry}

\usepackage[table]{xcolor}
\setlength{\parindent}{10pt}
\usepackage{amsfonts}
\usepackage{amsmath,amssymb}
% Choose a conveniently small page size
\usepackage{setspace}
% Using \doublespacing in the preamble 
% changes text to double line spacing
\onehalfspacing
\usepackage{listings}
\usepackage{color}
\title{Funzioni}
\author{Andrea Canale}

\begin{document}
	
\maketitle
\begin{spacing}{2.3}
	\tableofcontents
\end{spacing}
\section{Funzioni}
Una funzione con dominio A e codominio B è il sottoinsieme tra A e B ed è chiamato grafico della funzione:

$$ \Gamma \underline{\subset} AxB $$

Per definire una funzione usiamo la notazione: $ f:A \rightarrow B $

\section{Funzione ben definita}

Per verificare che una funzione $ f:A \rightarrow B $ sia ben definita dobbiamo controllare 2 cose:

\begin{itemize}
	\item La funzione sia \textbf{ben definita}, ossia che ogni elemento di A ha una sola immagine in B
	\item La funzione sia \textbf{funzionale}, ossia ogni elemento di A ha una sola immagine
\end{itemize}

\section{Immagine e controimmagine}

Data la funzione $ f:A \rightarrow B $, possiamo trovare l'immagine associata ad un valore del dominio.

\textbf{L'immagine} è quindi l'elemento associato nel codominio ad un elemento del dominio.

\textbf{La controimmagine} invece ci permette di eseguire l'operazione inversa: dato un elemento del codominio, la controimmagine di un elemento del codominio, è l'elemento nel dominio che restituisce quel valore.

L'immagine si può scrivere come $ f(a) = b $

La controimmagine si può scrivere come: $ f^{-1}(b) = a $

\section{Funzioni note}

\subsection{Funzione identità}
La funzione identità, è una funzione tale che $ id_A: A \rightarrow A $ e quindi quella funzione dove l'immagine e la controimmagine sono uguali. La funzione non modifica il suo argomento.

\subsection{Costante}
La funzione costante restituisce sempre lo stesso valore che identifichiamo con b.

\subsection{Successione}
Una successione di elementi in $ \mathbb{N} $ tale che $ s\left(0\right) = b_0, s\left(1\right) = b_1, s\left(2\right) = b_2... $

\section{Classificazione di funzioni}

\subsection{Funzioni suriettive}
Una funzione viene detta suriettiva se ogni elemento di B ha una controimmagine in A

\subsection{Funzioni iniettive}
Una funzione viene detta iniettiva se per ogni elemento dell'insieme A, non si hanno mai due immagini uguali.

$$ \forall a_1,a_2 \in A \text{ Allora } a_1 \neq a_2 $$

\subsection{Funzioni biettiva}
Una funzione viene detta biettiva se la funzione è sia suriettiva che iniettiva.

\section{Funzioni composte}
Possiamo anche definire una funzione che nasce dalla composizione di due o più funzioni. Ad esempio:
Date $ f: A \rightarrow B $ e $ g: B \rightarrow C $

$$ g \cdot f: A \rightarrow C $$

Possiamo comporle solo se il codominio di f coincide con il dominio di g.

Inoltre questa composizione deve essere ben definita.

\subsection{Proprietà associativa}
La composizione supporta la proprietà associativa, tale che:

$$ g \cdot f: A \rightarrow A = f \cdot g: B \rightarrow B $$

\textbf{La composizione non supporta la proprietà commutativa}

\subsection{Composizione di funzioni suriettive, iniettive e biettive}

\begin{itemize}
	\item Se f e g sono suriettive, $ g \cdot f $ è suriettiva
	\item Se f e g sono iniettive, allora $ g \cdot f $ è iniettiva
	\item Se f e g sono biettiva, allora $ g \cdot f $ è biettiva
\end{itemize}
Se non conosciamo le funzioni di partenza:

\begin{itemize}
	\item Se $ g \cdot f $ è suriettiva, allora g è suriettiva
	\item Se $ g \cdot f $ è iniettiva, allora f è iniettiva
\end{itemize}

\section{Funzioni invertibili}
Una funzione è invertibile, se e solo se, $ f:A \rightarrow B $ è biettiva.

L'inverso della funzione f è definito come: $ g \cdot f = id_A $ e $ f \cdot g = id_B $

Quindi la funzione inversa, dato il codominio ci restituisce il dominio mentre dato il dominio ci restituisce il codominio.
\end{document}