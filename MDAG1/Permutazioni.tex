\documentclass[a4paper, 10pt]{article}
\usepackage{geometry}

\usepackage[table]{xcolor}
\setlength{\parindent}{10pt}
\usepackage{amsfonts}
\usepackage{amsmath,amssymb}
% Choose a conveniently small page size
\usepackage{setspace}
% Using \doublespacing in the preamble 
% changes text to double line spacing
\onehalfspacing
\usepackage{listings}
\usepackage{color}


\title{Permutazioni}
\author{Andrea Canale}

\begin{document}
	\maketitle
	\tableofcontents

\section{Permutazioni}

Dato un insieme A, la sua permutazione è una funzione biettiva che cambia l'ordine degli elementi all'interno dell'insieme.

L'insieme delle permutazioni su n elementi è denotato come:

$$S_n=\{\text{permutazioni su S}\}$$

Dove $S$ è l'insieme che prende tutti i numeri da 0 a $n$.

\subsection{Rappresentare permutazioni}

Per rappresentare le funzioni usiamo una notazione matriciale di questo tipo:

$$ \begin{pmatrix}
	1 & 2 & 3 \\
	1 & 3 & 2 
\end{pmatrix}  $$

Dove nella prima riga abbiamo l'insieme originale e nella seconda riga la sua permutazione.

\section{Composizione di permutazioni}

L'insieme delle permutazioni $S_n$ ha un operazione di composizione definita come la composizione di funzioni, in quanto la permutazione è una funzione. 

$$\sigma \cdot \tau = a \rightarrow \sigma(\tau(a))$$

\subsection{Proprietà della composizione}

\begin{itemize}
	\item è associativa: $o	\sigma\ o\ \left(\tau\ o\ \nu\right)\ =\ \left(\sigma\ o\ \tau\right)o\ \nu$
	\item Ha un elemento neutro tale che $\sigma \cdot id = \sigma$
	\item Esiste un inverso(in quanto le permutazioni sono biettive) tale che $b \cdot b^{-1} = b^{-1} \cdot b = 1$
	\item Non vale la proprietà commutativa(come nella composizione di funzioni)
\end{itemize}

Le permutazioni formano un gruppo non commutativo.

\section{Permutazioni inverse}

Data la permutazione $$\sigma=\left(\begin{matrix}1\ &\ 2\ &\ 3\ &\ 4\ &\ 5\ \\3\ &\ 1\ &\ 4\ &\ 2\ &\ 5\ \\\end{matrix}\right)$$
Il suo inverso sarà:

$$\sigma^{-1}=\ \left(\begin{matrix}1\ &\ 2\ &\ 3\ &\ 4\ &\ 5\ \\2\ &\ 4\ &\ 1\ &\ 3\ &\ 5\ \\\end{matrix}\right)
$$

Da questo ne deduciamo che per calcolare l'inverso di una permutazione basta invertire le righe della permutazione di partenza e riordinarle.

\section{Cicli}

Una permutazione viene chiamata cicli se le immagini e le controimmagini della permutazioni combaciano. Cioè quanto ha una struttura del tipo:

$$\left(\begin{matrix}1\ &\ 2\ &\ i_1\ &\ k\ &\ i_2\ &\ n\ \\1\ &\ 2\ &\ i_2\ &\ k\ &\ i_1\ &\ n\ \\\end{matrix}\right)
$$

Ad esempio in $S_9$:

$$\sigma=\ \left(\begin{matrix}1\ &\ 2\ &\ 3\ &\ 4\ &\ 5\ &\ 6\ &\ 7\ & 8\ & 9\
	 \\1\ &\ 5\ &\ 7\ &\ 8\ &\ 3\ &\ 6\ &\ 4\ & 2\ & 9\ \\\end{matrix}\right)$$
	 
\subsection{Inverso di un ciclo}

L'inverso di un ciclo è il ciclo letto da destra verso sinistra. Ad esempio:

$\sigma=\left(3\ 5\ 2\right)$ diventa $\sigma^{-1}=\left(2\ 5\ 3\right)$

\section{Cicli disgiunti}

Due cicli sono disgiunti se la loro intersezione forma un insieme vuoto, o più semplicemente non hanno elementi in comune.

\subsection{Prodotto di cicli disgiunti}

La composizione di cicli disgiunti commuta.

Ogni permutazione diversa dall'identità può essere scritta come composizione di cicli disgiunti.

Esempio:

$$\left(\begin{matrix}1\ &\ 2\ &\ 3\ &\ 4\ &\ 5\ &\ 6\ &\ 7\ &\ 8\ &\ 9\ \\4\ &\ 5\ &\ 1\ &\ 6\ &\ 2\ &\ 3\ &\ 9\ &\ 8\ &\ 7\ \\\end{matrix}\right)\ 
$$ 
Può essere scritta come $\left(1\ 4\ 6\ 3\right)\ \left(2\ 5\right)\ \left(7\ 9\right)$

Da questo ne deduciamo che per scrivere una composizione come prodotto di cicli disgiunti "percorriamo" tutta la permutazione e quando troviamo che un elemento manda al primo chiudiamo il ciclo e riprendiamo il procedimento dal primo numero che manca.

\subsection{Tipo di una composizione di cicli disgiunti}

Data una composizione di k cicli disgiunti, il suo tipo è la k-upla che contiene la lunghezza di ogni cicli.

Esempio:

$\left(1\ 3\ 4\ 7\right)\ \left(2\ 5\right)$ ha tipo $(4, 2)$ oppure $(2, 4)\text{(forse)}$

\section{Prodotto di scambi}

Ogni permutazione si può scrivere come composizione di scambi. Questo deriva dal fatto che ogni ciclo di lunghezza l è composizione di l-1 composizioni.

Esempio:

$$c=\left(2\ 3\ 1\ 5\ 4\right)=\left(2\ 4\right)\left(2\ 5\right)\left(2\ 1\right)\left(2\ 3\right)$$

Inoltre questa scrittura non è univoca perchè aggiungendo un'identità, ad esempio $(2\ 3)$ otteniamo lo stesso risultato.

Notiamo che le trasposizioni che intervengono non sono disgiunte e questo comporta che la commutatività non vale.

Questa notazione viene usata per risparmiare spazio, infatti le permutazioni di $S_n$ sono $n!$ mentre il prodotto di scambi è $ \begin{pmatrix}
	n \\
	2 
\end{pmatrix}  $

\section{Parità di permutazioni}

Le permutazioni sono classificate in base al numero di scambi che la compongono.

\begin{itemize}
	\item Le permutazioni pari si scrivono con un numero pari di trasposizioni
	\item Le permutazioni dispari si scrivono con un numero dispari di disposizioni
\end{itemize}

La cardinalità dell'insieme delle permutazioni pari e dispari è $\frac{n!}{2}$

\subsection{Parità di cicli}

\begin{itemize}
	\item Un l-ciclo è pari se l è dispari
	\item Un l-ciclo è dispari se l è pari
\end{itemize}

Questo viene dal teorema precedente dove imponevamo che un ciclo si può scrivere come prodotto di l-1 trasposizioni.

\subsection{Parità di una composizione di cicli disgiunti}

La parità di una composizione di cicli disgiunti è data dalla somma di tutte le lunghezze dei singoli cicli. Se questa somma è pari allora la permutazione è pari, altrimenti è dispari.

\section{Periodo di una permutazione}

Data una permutazione $\sigma$, il periodo è il minimo numero intero $n > 0$ tale che $\sigma^n=(1)$

Questo numero è scritto come $per(\sigma)$

\subsection{Periodi noti}

\subsubsection{Periodo di un k-ciclo}

Il periodo di un k-ciclo è k.

\subsubsection{Periodo di una trasposizione}

Il periodo di una trasposizione è sempre uguale al numero di componenti che la compongono.

Ad esempio $per((1\ 2)) = 2$

\subsubsection{Periodo di un prodotto di cicli disgiunti}

Il periodo di un prodotto di cicli disgiunti è dato dal minimo comune multiplo delle lunghezze dei cicli.

Ad esempio: $\sigma$ ha tipo $(l_1\ ... \ l_k)$ allora $per(\sigma) = mcm(l_1\ ... \ l_k)$

\subsubsection{Periodo di una composizione di trasposizioni}

Questo non si può calcolare direttamente ma bisogna sempre riscrivere la composizione come prodotto di cicli disgiunti.

\subsection{Calcolo di potenze n-esime}

Data una permutazione $\sigma$, se vogliamo calcolare $\sigma^n$ possiamo procedere seguendo questi passi:

\begin{itemize}
	\item Trovare $per(\sigma) = p$
	\item Calcolare $\frac{n}{p}$(compreso il resto $r$)
	\item Scrivere $\sigma^n = \sigma^{q \cdot p + r} = (\sigma^p)^q \cdot \sigma^r$
	\item Ora possiamo semplificare $(\sigma^p)^q$ perchè sappiamo che è uguale all'identità $(1)$
	\item Calcoliamo solo $\sigma^r$
\end{itemize}
\end{document}