\documentclass[a4paper, 10pt]{article}
\usepackage{geometry}

\usepackage[table]{xcolor}
\setlength{\parindent}{10pt}
\usepackage{amsfonts}
\usepackage{amsmath,amssymb}
% Choose a conveniently small page size
\usepackage{setspace}
% Using \doublespacing in the preamble 
% changes text to double line spacing
\onehalfspacing
\usepackage{listings}
\usepackage{color}
\title{Aritmetica Modulare}
\author{Andrea Canale}

\begin{document}
	\maketitle
	\tableofcontents
	
\section{Congruenza}

Dato un numero $N \geq 2$, chiamato modulo, e due numeri $m,n \in \mathbb{Z}$, questi numeri si dicono congruenti a modulo $N$ se:
$$N\ |\ m - n$$

E lo scriviamo come $$m \equiv n\ mod\ N$$

\textbf{Notiamo che la congruenza è una relazione di equivalenza}

\section{Classi di equivalenza}

Una classe di equivalenza su un numero è definita come segue:
$$[a]_N=\{b \in \mathbb{Z} | a \equiv b\ mod\ N \}$$

Alternativamente, se viene esplicitato il modulo si può scrivere come: $\overline{a}$

Questo tipo di classe si dice classe di resto, per calcolarle più facilmente si può scrivere: 
$$[a]_N = \{a + kN\ |\ k \in \mathbb{Z}\} = a + N\mathbb{Z}$$

Notiamo che le classi di resto hanno una certa ciclicità, ad esempio se $N=5$, sappiamo che $[5]_5 = [0]_5$ e così via.

Inoltre le classi di resto di $N$ sono esattamente $N$

\subsection{Rappresentante canonico}

Sia $[a]_N$ una classe d'equivalenza, allora $[a]_N = [r]_N$ dove r è il resto della divisione $a:N$. Questa classe d'equivalenza viene detta rappresentante canonico.

\section{Insieme delle classi di resto $\mathbb{Z_N}$}

L'insieme di tutte le classi di resto di un determinato modulo viene denotato come: $$\mathbb{Z}_N = \{[0]_N, ..., [N-1]_N\}$$

Tutte le classi di resto formano un ricoprimento di $\mathbb{Z}$

\subsection{Operazioni su $\mathbb{Z_N}$}

\subsubsection{Addizione}

L'addizione tra classi di resto viene definita come: $\overline{a} + \overline{b} = \overline{a+b}$

Inoltre valgono l'associatività e la commutatività.

L'elemento neutro è $\overline{0}$. L'inverso è $\overline{-a}$

\subsubsection{Moltiplicazione}

La moltiplicazione tra classi di resti è definita come: $\overline{a} \cdot \overline{b} = \overline{a \cdot b}$.

Valgono l'associatività, la commutatività e la distributività.

La moltiplicazione ha come elemento neutro $\overline{1}$. L'esistenza dell'inverso non è garantita.

\section{Invertibilità di classi}

L'insieme delle classi invertibili è definito come $$\mathbb{Z}_n^* = \{\overline{a} \in \mathbb{Z}_n | \exists \overline{b} \in \mathbb{Z}_n \text{ t.c } \overline{a} \cdot \overline{b} = \overline{1}\}$$

La cardinalità di questo insieme è definita dalla funzione di Eulero. Tuttavia questa funzione non ha un andamento prevedibile.

\subsection{Calcolare l'invertibilità di una classe}

$[a]_N$ è invertibile se e solo se $a$ è coprimo con il modulo.

\subsection{Calcolare il valore della funzione di Eulero}

Per i numeri primi, la funzione di Eulero si calcola come: $$\varphi(p) = p - 1$$

Oppure per le potenze di numeri primi: $$\varphi(p^n) = p^{n - 1}(p-1) $$

\textbf{Da ciò ne ricaviamo che per calcolare la funzione di Eulero per un generico numero $n$, dobbiamo prima scomporre in fattori primi e poi usare queste due formule per calcolare il risultato, moltiplicando tra loro i valori delle funzioni di Eulero per le singole funzioni.}

Notiamo inoltre che se $MCD(a,b) = 1$, allora $$\varphi(ab) = \varphi(a)\varphi(b)$$
% classi di resto invertibili

\section{Zero divisori}

Una zero divisori in $\mathbb{Z}_N$ è una classe $\overline{a} \neq \overline{0}$ tale che $\overline{a} \cdot \overline{b} = \overline{0}$ per qualche $\overline{b} \neq \overline{0}$

$a \in \mathbb{Z}_N$ è uno zero divisore se e solo se $(a,N) > 1$

\section{Generatori}

Una classe ne genera un'altra se il modulo e il fattore sono coprimi.

\section{Congruenze lineare}

Una congruenza lineare è un'equivazione della forma $aX \equiv c\ mod\ N$ dove x è l'incognita che vogliamo risolvere.

Una congruenza di questo tipo è risolvibile se e solo se $d=MCD(a,N)$ divide $c$.

Abbiamo due casi:

\begin{itemize}
	\item $MCD(a,N) = 1$, in questo caso sappiamo che esiste una soluzione
	\item $MCD(a,N) = d$, sappiamo che esistono $d$ soluzioni in $mod\ N$ 
\end{itemize}

\subsection{Trovare le soluzioni}

Per trovare le soluzioni dobbiamo trovare l'identità di Bezout per (a,N) e il risultato sarà il coefficiente diverso da N


\end{document}