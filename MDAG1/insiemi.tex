\documentclass[a4paper, 10pt]{article}
\usepackage{geometry}

\usepackage[table]{xcolor}
\setlength{\parindent}{10pt}
\usepackage{amsfonts}
\usepackage{amsmath,amssymb}
% Choose a conveniently small page size
\usepackage{setspace}
% Using \doublespacing in the preamble 
% changes text to double line spacing
\onehalfspacing
\usepackage{listings}
\usepackage{color}

\title{Insiemi}
\author{Andrea Canale}

\begin{document}

\maketitle
\tableofcontents

\section{Definizione insieme}

Un insieme è una collezione ben definita di oggetti

Gli elementi in un insieme sono univoci

Può essere identificato come $ A = \{1, 2, 3, ...\} $
Possiamo anche specificare una condizione: $ A = \{ x \in \mathbb{Q} | x^2 - 1 = 0 \} $
\section{Simboli degli insiemi}

\subsection{Inclusione}
$ \in $ , L'insieme include, può essere usato solo per gli elementi

La sua negazione è $ \notin $

\subsection{Qualificatori universali}

\subsubsection{Per ogni}
$ \forall $, indica che tutti gli elementi di un insieme rispettano una condizione

\subsubsection{Esiste}
$ \exists $, indica che esiste almeno un elemento nell'insieme che rispetta una condizione

\subsubsection{Negazione}
$ \tilde{} $, indica il negato di una condizione

\section{Cardinalità di un insieme}
La cardinalità di un insieme, indicata come $ |A| = n $, è il numero di elementi che un insieme contiene

Un insieme senza elementi è detto vuoto ed è indicato come Ø = {} che è diverso da A={Ø} che contiene un elemento

\section{Sottoinsiemi}
I sottoinsiemi di un insieme A sono gli insiemi costituiti dagli elementi di A.

Un insieme B è sottoinsieme di A se ogni suo elemento è presente anche in A.

La relazione che lega A e B si può scrivere così: $ B \subset A $

Esistono 3 tipologie di insiemi:

\begin{itemize}
	\item Sottoinsiemi banali, Ø e l'insieme stesso
	\item Sottoinsiemi proprio, dove $ B \subset A $ con $ B \neq A $
\end{itemize}

\subsection{Insieme delle parti}
L'insieme delle parti è definito da tutti i sottoinsiemi di A.
$$ P(A) = \{ B | B \subset A \} $$

Ad esempio con $ A = \{ a, b\} $, l'insieme delle parti è
$$ P(A) = \{ \emptyset, {a}, {b}, A\} $$

Due insiemi sono uguali se $ A \subseteq B $  e $ B \subseteq A $ e $ P(A) = P(B) $

\subsection{Intersezione e unione}

\subsubsection{Intersezione $ \cap $}
Dati A e B, $ A \cap B = \{ x | x \in A, x \in B \} $

\subsubsection{Unione $ \cup $}
Dati A e B, $ A \cup B = \{ x | x \in A, x \in B \} $

\section{Proprietà degli insiemi}

\subsection{Associatività}
$$ (A \cap B) \cap C = A \cap (B \cap C) $$

\subsection{Idenpotente}
Un insieme intersecato o unito per un insieme vuoto, dà un insieme vuoto.

\subsection{Distribuibilità}
$$ A \cap (B \cup C) = (A \cap B) \cup (A \cup C) $$
$$ A \cup (B \cap C) = (A \cup B) \cap (A \cup C) $$

\section{Differenza tra 2 insiemi}

Siano A e B due insiemi finiti, la loro differenza è denotata come: 

$$ A \backslash B = \{x \in A | x \notin B\} $$

\subsection{Complementare}

Nel caso in cui A sia sottoinsieme di B, l'insieme formato dalla loro differenza è detto complementare ed è definito come: $ C_b(A) = B \backslash A $
\section{Leggi di De Morgan}

Dati 3 insiemi A, B e C, valgono le seguenti regole:

$$ C_x(A \cap B) = C_x(A) \cup C_x(B) $$
$$ C_x(A \cup B) = C_x(A) \cap C_x(B) $$

\section{Ricoprimento}

Dato un insieme A e una famiglia di sottoinsiemi $ X = \{Ai\}_{i \in X} $ , il suo ricoprimento è definito come la famiglia di sottoinsiemi che uniti danno l'insieme A di partenza. è definito attraverso questa notazione:

$$ \bigcup_{i \in X}A_i = X $$

\section{Partizioni}

Dato un insieme A e una famiglia di sottoinsiemi $ X = \{Ai\}_{i \in X} $ che sono ricoprimento di un insieme, essi sono una partizione di un insieme se soddisfano 3 requisiti:

\begin{itemize}
	\item Sono ricoprimento di A
	\item $ \forall i \in X, A_i \neq\emptyset $
	\item $ \forall i, j \in X $ tali che $ i \neq j $ i sottoinsiemi Ai e Aj sono disgiunti, $ A_i \cap A_j  = \emptyset $. Ossia i sottoinsiemi sono diversi tra loro
\end{itemize}

\section{Prodotto cartesiano}

Dati due insiemi A e B, il loro prodotto cartesiano e definito da un insieme che ha come elementi la coppia di un elemento dell'insieme A e un elemento dell'insieme B:

$$ A X B = \{(a,b) | a \in A, b \in B\} $$

La cardinalità di questo insieme A X B e definita come: $ |AXB| = |A| \cdot |B| $

Il prodotto cartesiano viene usato per esprimere relazioni tra insiemi.

\section{Proprietà delle relazioni}

\begin{itemize}
	\item Dato un elemento di A X B, esso è veramente una relazione se $ (a, b) \in \mathbb{R} $
	\item Una relazione viene detta riflessiva se $ \forall a \in A$, a è in relazione con se stesso, quindi $ a \in \mathbb{R} x \mathbb{R} $
	\item Relazione simmetrica
	\item Transitiva
	\item Equivalente, se è riflessiva, simmetrica e transitiva
	\item Dato un insieme A, esiste una relazione tra l'insieme delle relazioni di equivalenza e la sua partizione
\end{itemize}



\end{document}