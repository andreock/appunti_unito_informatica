\documentclass[a4paper, 10pt]{article}
\usepackage{geometry}

\usepackage{graphicx}
\graphicspath{ {./circuiti/} }
\usepackage[table]{xcolor}
\setlength{\parindent}{10pt}
\usepackage{amsfonts}
\usepackage{amsmath,amssymb}
% Choose a conveniently small page size
\usepackage{setspace}
% Using \doublespacing in the preamble 
% changes text to double line spacing
\onehalfspacing
\usepackage{listings}
\usepackage{color}

\definecolor{dkgreen}{rgb}{0,0.6,0}
\definecolor{gray}{rgb}{0.5,0.5,0.5}
\definecolor{mauve}{rgb}{0.58,0,0.82}

\lstset{frame=tb,
	language=C,
	aboveskip=3mm,
	belowskip=3mm,
	showstringspaces=false,
	columns=flexible,
	basicstyle={\small\ttfamily},
	numbers=none,
	numberstyle=\tiny\color{gray},
	keywordstyle=\color{blue},
	commentstyle=\color{dkgreen},
	stringstyle=\color{mauve},
	breaklines=true,
	breakatwhitespace=true,
	tabsize=3
}


\title{Strutture algebriche}
\author{Andrea Canale}

\begin{document}
	\maketitle
	\tableofcontents
	
\section{Strutture algebriche}

Una struttura algebrica è un insieme sul quale sono definite delle operazioni binarie(noi ne tratteremo solo una).

Una struttura algebrica può essere di tre tipi:

\begin{itemize}
	\item Semigruppo se l'operazione è associativa
	\item Monoide se l'operazione è associativa ed esiste un elemento neutro
	\item Gruppo se l'operazione è associativa, esiste elemento neutro ed esiste un inverso $y$ tale che $x \cdot y = y \cdot x = e$
\end{itemize}

Se l'operazione commuta, la struttura si dice commutativa o abeliana(solo nel caso di gruppi).

\textbf{L'operazione deve essere ben definita per tutti gli elementi della struttura algebrica(il dominio e il codominio devono coincidere con l'insieme nella struttura)}

Un sottoinsieme $Y \subset X$ si dice stabile o chiuso rispetto a $*$ se $\forall a, b \in Y$, $a * b \in Y$

\section{Strutture algebriche note}

\subsection{Strutture algebriche su numeri}
\begin{itemize}
	\item $(\mathbb{N} \backslash 0,+)$ È un semigruppo commutativo. La somma è associativa ma manca l’elemento neutro. La somma è commutativa
	\item $(\mathbb{N} ,+)$ È un monoide commutativo. Infatti manca l’inverso
	\item $(\mathbb{Z},+)$ È un gruppo abeliano. Esiste l’opposto.
	\item $(\mathbb{Z}_n,+)$ È un gruppo abeliano.
	\item $(\mathbb{Z}_n, \cdot)$ È un monoide commutativo. Non sempre esiste l’inverso
\end{itemize}

Strutture algebriche su insiemi
Sia $ Y \neq \emptyset, X=P(Y)$

La struttura $(X, \cup)$
\begin{itemize}
	\item È associativa
	\item Ha elemento neutro $\emptyset$
	\item Non ha un inverso
\end{itemize}

La struttura $(X, \cap)$
\begin{itemize}
	\item È associativa
	\item Ha elemento neutro $Y$
	\item Non ha un inverso
\end{itemize}

Sono entrambi dei monoidi commutativi


\subsection{Monoide delle parole(o monoide libero su A)}

A partire da un alfabeto(insieme non vuoto. $P=\{\text{stringhe finite su A}\}$
	
Definiamo su P l’operazione di concatenazione di parole: $x_1,...,x_n*y_1,...,y_n=x_1,...,(x_ny)_1,...,y_n$


$(P,*)$ È una struttura algebrica:

\begin{itemize}
	\item È associativa
	\item Ha elemento neutro $\lambda$
	\item Non ha un inverso
\end{itemize}

Concludiamo che è un monoide non commutativo.


\subsection{Gruppo delle permutazioni su X}

Dato il gruppo delle funzioni biettive su $X: X={f \in f(x) | \text{f biettiva}})$
È un gruppo non abeliano. È anche chiamato gruppo simmetrico su X. Notiamo che è l’insieme delle permutazione su X.

\subsection{Gruppo delle matrici quadrate}

Dato l’insieme $X={a \in M(n,K) |deta \neq 0}),$ è detto gruppo lineare di ordine n a coefficienti in K.
È un gruppo non commutativo se $n\geq2$.

\section{Notazione moltiplicativa e additiva}

La notazione moltiplicativa si usa per gruppi del tipo $(G, *)$:

\begin{itemize}
	\item $a*b = ab$
	\item $a^n = a \cdot a \cdot ... \cdot a$
	\item $a^{-1}$ è l'inverso di $a$
\end{itemize}

La notazione additiva si usa per gruppi del tipo $(G, +)$:

\begin{itemize}
	\item $a*b = a + b$
	\item $na = a + a + ... + a$
	\item $-a$ è l'inverso di $a$
\end{itemize}

\section{Gruppi}

\subsection{Proprietà di un gruppo}

Dato un gruppo $G$, valgono le seguenti proprietà:
\begin{itemize}
	\item L'elemento neutro è unico
	\item L'inverso di un elemento è unico
	\item $(x * y)^{-1} = y^{-1} \cdot x^{-1}$
	\item Valgono le leggi di cancellazione destra e sinistra: $x * y = x * z$ si può semplificare $y = z$
\end{itemize}

Se $G$ è finito, la sua cardinalità si dice ordine di $G$

\subsection{Prodotto diretto tra gruppi}

Dati due gruppi $G_1$ e $G_2$ possiamo costruire il prodotto scalare tra due gruppi e definirlo come prodotto diretto tra gruppi:
$$(a_1, b_1) * (a_2, b_2) = (a_1 + a_2, b_1 + b_2)$$
Dove $(a_1, b_1) \in G_1$ e $(a_2, b_2) \in G_2$ 

Il risultato di $G_1\ x\ G_2$ sarà anch'esso un gruppo. Inoltre, il prodotto diretto tra gruppi abeliani è anch'esso abeliano.

Notiamo che $G^N = G\ x\ G\ x\ G\ ...$

\section{Sottogruppi}

Dato un gruppo $(G, *)$, un sottogruppo $H$ di G è un sottoinsieme di $G$ tale che $(H, *)$ è un gruppo. Viene indicato come: $H \leq G$

Un sottogruppo deve rispettare tre proprietà:

\begin{itemize}
	\item $H$ stabile rispetto a $*$
	\item $H$ ha lo stesso elemento neutro di $G$
	\item $H$ contiene l'inverso di ogni suo elemento e coincide con quello di $G$
\end{itemize}

Notiamo che un gruppo $G$ ha sempre almeno due sottogruppi detti sottogruppi banali: $G$ e $\{e\}$

Inoltre l'intersezione tra sottogruppi è un sottogruppo tale che $H_1, H_2 \leq G$, allora $H_1 \cap H_2 \leq G$
\subsubsection{Sottogruppi di $\mathbb{Z}$}

Tutti i sottogruppi di $\mathbb{Z}$ sono del tipo $N\mathbb{Z}$ per qualche $N \in \mathbb{N}$

Inoltre: $N\mathbb{Z} \cap M\mathbb{Z} = m\mathbb{Z}$ dove $m = mcm(N,M)$

\subsection{Laterali}

Dato un gruppo $G$ e un suo sottogruppo $H \leq G$, definiamo le relazioni: 

\begin{itemize}
	\item $X \sim_1 Y $ Se $xy^{-1} \in H$
	\item $X \sim_2 Y $ Se $y^{-1}x \in H$
\end{itemize}

Queste due relazioni sono relazioni di equivalenza su $G$.

Sia $G$ un gruppo e $H \leq G$. Dato un elemento $g \in G$ detto \textbf{rappresentante del laterale} si dice:

\begin{itemize}
	\item Laterale sinistro di $H$ il sottoinsieme: $gH = \{gh | h \in H\} \subset G$
	\item Laterale destro di $H$ il sottoinsieme: $Hg = \{hg | h \in H\} \subset G$
\end{itemize}

Esempio:

$G=S_3=\{(1), (1\ 2). (1\ 3), (2\ 3), (1\ 2\ 3), (1\ 3\ 2)\}$
$H=\{(1), (1\ 2)\}$

I laterali destri saranno della forma: 

$H(1) = \{(1), (1\ 2)\} = H(1\ 2)$

$H(1\ 3) = \{(1)(1\ 3), (1\ 2)(1\ 3)\} = \{(1\ 3), (1\ 3\ 2)\} = H(1\ 3\ 2)$

E così via...

Per i laterali sinistri vale la stessa regola ma si inverte la moltiplicazione

$(1)H = \{(1), (1\ 2)\} = (1\ 2)H$

$(1\ 3)H = \{(1\ 3)(1), (1\ 3)(1\ 2)\} = \{(1\ 3), (1\ 2\ 3)\} = (1\ 2\ 3)H$

\textbf{Se $G$ è commutatitivo i laterali destri e sinistri coincidono.}

Valgono queste proprietà per i laterali destri e sinistri:

\begin{itemize}
	\item La funzione $\theta: H \rightarrow Hx$ che manda $h \rightarrow hx$(e il suo analogo sinistro) è biettiva
	\item Se $G$ è finito, tutti i laterali destri e sinistri hanno la stessa cardinalità.
\end{itemize}

\subsection{Teorema di Lagrange}

Sia $G$ un gruppo finito e $H \leq G$. L'ordine di H divide l'ordine di G.

Notiamo che il teorema di Lagrange non si inverte tranne per i gruppi abeliani.

\section{Omomorfismi}

Siano $(G_1, *_1)$ e $(G_2, *_2)$ gruppi, una funzione $f: G_1 \rightarrow G_2$ si dice omomorfismo se:$$f(x *_1 y) = f(x) *_2 f(y)$$

\subsection{Omomorfismi ben definiti}

Un omomorfismo deve essere ben definito, ad esempio:

$f: \mathbb{Z}_8 \rightarrow \mathbb{Z}_{12}$

Se prendiamo $[1]_8 = [9]_8$, allora $f(1) = f(9)$ 

\textbf{Quindi deve mantenere le proprietà del gruppo}

In generale se $a = b$ allora $f(a) = f(b)$

\subsection{Proprietà degli omomorfismi}

Dato un omomorfismo $f: G_1 \rightarrow G_2$, valgono le seguenti proprietà:

\begin{itemize}
	\item $f(eG_1) = eG_2$
	\item $f(x^{-1}) = f(x)^{-1}$ $\forall x \in G_1$
	\item $f(x^n) = f(x)^n$
	\item Dato un sottogruppo $H$ di $G_1$, allora $f(H) \leq G_2$ dove $f(H)$ è la funzione applicata su tutti gli elementi di $H$
	\item Dato un sottogruppo $H$ di $G_2$, allora $f^{-1}(H) \leq G_1$
\end{itemize}

Se non valgono queste proprietà, la funzione non è omomorfismo.

\subsection{Immagini e Nucleo}

Dato un omomorfismo $f: G_1 \rightarrow G_2$:

L'insieme definito da $f(G_1)$ è detto immagini della funzione ed è denotato come $im(f)$

L'insieme definito come $\{x \in G_1 | f(x) = eG_2\}$ è detto nucleo ed è denotato come $ker(f)$

Un omomorfismo $f$ è iniettivo, se e solo se, $ker(f) = \{eG_1\}$

\subsection{Classificazione di omomorfismi}

\begin{itemize}
	\item \textbf{Monomorfismo} se è un omomorfismo iniettivo
	\item \textbf{Epimorfismo} se è suriettivo
	\item \textbf{Isomorfismo} se è biettivo
	\item \textbf{Endomorfismo} se $G_1 = G_2$(cioè f agisce sullo stesso gruppo)
	\item \textbf{Automorfismo} se è endomorfismo e monomorfismo
\end{itemize}

\section{Isomorfismi}

Due gruppi si dicono isomorfi se esiste un isomorfismo $f: G_1 \rightarrow G_2$. Viene indicato come $G_1 \simeq G_2$

Due gruppi isomorfi hanno le stesse proprietà, ad esempio: se $G_1$ è finito, lo sarà anche $G_2$ oppure se $G_1$ è abeliano lo è anche $G_2$, ecc...

\section{Gruppi ciclici}

Sia $(G_1, *)$ un gruppo, possiamo facilmente costruire sottogruppi di $G_1$ a partire da un elemento $g \in G_1$, definiamo un sottogruppo ciclico generato da x come: $$\textlangle x \textrangle = \{g^n \in G_1| n \in \mathbb{N}\}$$

Notiamo che questa notazione vale per gruppi scritti come notazione moltiplicativa, per gruppi in notazione additiva abbiamo: $$\textlangle x \textrangle = \{ng \in G_1| n \in \mathbb{N}\}$$

Se esiste un elemento $x$ per il quale $\textlangle x \textrangle = G_1$, il gruppo $G_1$ si dice ciclico e $x$ è il generatore di $G_1$

Notiamo inoltre che se $G$ è un gruppo ciclico, allora $G$ è abeliano. Da ciò ne ricaviamo che qualsiasi gruppo non abeliano sicuramente non è ciclico. Tuttavia non tutti i gruppi abeliani sono ciclici.

Osserviamo anche che il prodotto di gruppi ciclici non è necessariamente un gruppo ciclico.

\subsection{Generatori di $\mathbb{Z}_N$}

I generatori del gruppo $Z_N$ coincidono con gli elementi invertibili $\mathbb{Z}_N^*$

\subsection{Gruppi finiti di ordine primo}

Se l'ordine di un gruppo $|G|=p$ è un numero primo e il gruppo è generato da $x$, per il teorema di Lagrange $|\textlangle x \textrangle| = d$ divide $p$, tuttavia ciò è possibile solo se $d = 1$ o $d = p$. Se noi imponiamo $x \neq e$ otteniamo che l'unica possibilità è $d = p$. 

Quindi se $G$ ha un ordine primo, allora è generato da qualsiasi $x \neq e$. Concludiamo che $G$ è un gruppo ciclico generato da qualsiasi suo elemento $\neq e$

Per quello che abbiamo detto prima, tutti i gruppi di ordine primo sono abeliani.
\subsection{Teorema di classificazione di gruppi ciclici}

Se G è un gruppo ciclico, allora:

\begin{itemize}
	\item Se G è infinito, allora $G$ è isomorfo a $\mathbb{Z}$
	\item Se G è finito con ordine $|G| = n$, allora $G \simeq \mathbb{Z}_N$
\end{itemize}

\subsection{Periodo di un gruppo ciclico}

Dato un gruppo $G$ ciclico e un elemento(anche non generatore) $x \in G$, definiamo il suo periodo come: $$per(x) = |\textlangle x \textrangle|$$

Osserviamo che in generale, se $|\textlangle x \textrangle|$ ha periodo finito, allora il periodo è il minimo numero $n$ tale per cui $x^n = e$.

Inoltre per il teorema di Lagrange, se $per(x)$ divide l'ordine di $G$

Ciò ci viene utile per risolvere potenze molto alte: poniamo $|G| = per(x) \cdot k$. Per risolvere $x^n$ possiamo scrivere $x^{d \cdot k} = (x^d)^k$. Tuttavia $x^d=e$ quindi $x^n=e$.

Questo perchè: \textbf{Ogni elemento di un gruppo finito elevato all'ordine del gruppo dà l'elemento neutro}

Inoltre, l'unico elemento con periodo 1 è l'elemento neutro
\\

Questo ha un applicazione particolare per il calcolo di potenze $\mathbb{Z}_N^*$.

Dato che in $\mathbb{Z}_N^*$, $|\mathbb{Z}_N^*| = \varphi(N)$, Otteniamo che $\overline{a}^{\varphi(N)} = \overline{1}$ se $(a,N) = 1$

\subsection{Piccolo teorema di Fermat}

Nel caso particolare in cui il modulo $N$ sia un numero primo, si ha il piccolo teorema di Fermat: $$p \nmid a \text { allora } a^{p-1} \equiv 1 mod p$$
\subsection{Criteri di ciclicità di prodotti ciclici}

Dato due gruppi ciclici $G_1$ e $G_2$, $G_1\ x\ G_2$ è ciclico, se e solo se gli ordini di $G_1$ e $G_2$ sono coprimi.
Notiamo che $|G_1\ x\ G_2| = |G_1| \cdot |G_2|$
\end{document}