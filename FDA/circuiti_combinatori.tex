\documentclass[a4paper]{article}
\usepackage{geometry}

\usepackage{graphicx}
\graphicspath{ {./circuiti/} }
\usepackage[table]{xcolor}
\setlength{\parindent}{10pt}
\usepackage{amsfonts}
\usepackage{amsmath,amssymb}
% Choose a conveniently small page size
\usepackage{setspace}
% Using \doublespacing in the preamble 
% changes text to double line spacing
\doublespacing
\usepackage{listings}
\usepackage{color}

\definecolor{dkgreen}{rgb}{0,0.6,0}
\definecolor{gray}{rgb}{0.5,0.5,0.5}
\definecolor{mauve}{rgb}{0.58,0,0.82}

\lstset{frame=tb,
	language=C,
	aboveskip=3mm,
	belowskip=3mm,
	showstringspaces=false,
	columns=flexible,
	basicstyle={\small\ttfamily},
	numbers=none,
	numberstyle=\tiny\color{gray},
	keywordstyle=\color{blue},
	commentstyle=\color{dkgreen},
	stringstyle=\color{mauve},
	breaklines=true,
	breakatwhitespace=true,
	tabsize=3
}


\title{Circuiti combinatori}
\author{Andrea Canale}

\begin{document}
	\maketitle
	\tableofcontents

\section{Porte logiche}

% TODO: mettere foto porte logiche
\section{Algebre di Boole}

Dato un insieme S contenente elementi distinti 0 e 1 e due operazioni binarie $+$ e $\cdot $, questo insieme è un algebra di Boole se valgono le seguenti proprietà:

\begin{itemize}
	\item Associatività: $(x+y)+z=x+(y+z)$ e $(x \cdot y) \cdot z=x \cdot (y \cdot z)$
	\item Commutatività: $x+y=y+x$ e $x \cdot y = y \cdot x$
	\item Distributività: $x \cdot (y+z) = (x \cdot y) + (x \cdot z)$ e $ x + (y \cdot z) = (x+y) \cdot (x+z)$ e $x + (y \cdot z) = (x + y) \cdot (x + z)$
	\item Esistenza di elementi neutri: $x+0=x$ e $x \cdot 1 = x$
	\item Esistenza di un inverso: $x+x^{'} = 1$ e $x \cdot x^{'} = 0$
\end{itemize}

L'inverso in un algebra di Boole è univoco tale che:

$$ x+y = 1 \text{ e } xy=0 \text{ allora } y=x^{'}$$

Inoltre soddisfa altre 6 proprietà:

\begin{itemize}
	\item Idempotenza: $x+x=x$ e $xx=x$
	\item Legge del limite/dominanza: $x+1=1$ e $x \cdot 0 = 0$
	\item Legge di assorbimento: $x+xy=x$ e $x(x+y) = x$
	\item Legge d'involuzione: $(x^{'})^{'} = x$
	\item Legge 0 e 1: $0^{'} = 1$ e $1^{'} = 0$
	\item Leggi di De Morgan per le algebre di Boole: $(x+y)^{'} = x^{'} \cdot y^{'}$ e $(xy)^{'} = x^{'} + y^{'}$
\end{itemize}

\section{Dualità}

Il duale di un espressione booleana è l'espressione ottenuta invertendo tutte le operazioni che incontriamo:
Rimpiazziamo 0 con 1 e 1 con 0 e Rimpiazziamo + con $\cdot$ e $\cdot$ con $+$

Notiamo che il duale di un teorema su un algebra di Boole è sempre un teorema.

\section{Funzioni booleane}

Una funzione booleana è una funzione di n argomenti su un algebra di Boole. è una funzione della forma:

$$f(x_1, ..., x_n) = X(x_1, ..., x_n)$$

Ad esempio: $f(x_1, x_2) = x_1 \land \overline{x_2}$

\subsection{Ricavare funzioni da tavole di verità}

Partendo da una tavola della verità del tipo: 
$$
	\begin{array}{|c c c|c|}
		x_1 & x_2 & x_3 & f(x_1, x_2, x_3)\\
		\hline
		1 & 1 & 1 & 1\\
		1 & 1 & 0 & 0\\
		1 & 0 & 1 & 0\\
		1 & 0 & 0& 1\\
		0 & 1 & 1 & 0\\
		0 & 1 & 0 & 1\\
		0 & 0 & 1 & 0\\
		0 & 0 & 0 & 0\\
	\end{array}
$$

Procediamo così:

\begin{itemize}
	\item Scriviamo le formule che danno come output 1. Queste formule sono dette mintermini
	\item Disgiungiamole tra di loro
\end{itemize}

In questo caso abbiamo:
\begin{itemize}
	\item $x_1 x_2 x_3$
	\item $x_1 \overline{x_2} \overline{x_3}$
	\item $\overline{x_1} x_2 \overline{x_3}$
\end{itemize}

Otteniamo che la seguente tavola di verità si ottiene dalla funzione: $$f(x_1, x_2, x_3) = (x_1 x_2 x_3) \lor (x_1 \overline{x_2} \overline{x_3}) \lor \overline{x_1} x_2 \overline{x_3}$$

Questa forma per una funzione booleana è dette \textbf{forma disgiuntiva normale}.

\section{Porte funzionalmente complete}

Qualsiasi funzione booleana si può scrivere come risultato di combinazioni di $\{AND, OR, NOT\}$. Ne deduciamo che queste porte sono funzionalmente complete.

In particolare $\{AND, NOT\}$ e $\{OR, NOT\}$ sono funzionalmente complete.

Anche la porta NAND è funzionalmente completa.

\section{Semplificazione di circuiti combinatori}

Possiamo semplificare circuiti combinatori attraverso le leggi di Quine-McCluskey:

\begin{itemize}
	\item $Ea \lor E\overline{a} = E(a \lor \overline{a} = E1 = E$
	\item $E=E\lor Ea$
\end{itemize}

Dove E è una condizione booleana. Queste regole valgono per qualsiasi algebra di boole

\section{Esempi di circuiti combinatori}

\subsection{Half-Adder}

\includegraphics[scale=0.6]{half_adder}

Un half-adder somma i due ingressi $x,y$ e dà il risultato in uscita su $s$ e l'eventuale riporto in $c$.

$$
\begin{array}{|c c |c|c|}
	x & y & c & s\\
	\hline
	1 & 1 & 1 & 0\\
	1 & 0 & 0 & 1\\
	0 & 1 & 0 & 1\\
	0 & 0 & 0& 0\\
\end{array}
$$

\subsection{Full-Adder}

Un full-adder combina vari half-adder per fare somme a 3 o più bit.

\includegraphics[scale=0.6]{full_adder}

\subsection{Decoder}

Prende in ingresso n bit e attiva l'uscita sulla base dei bit in ingresso convertiti in numero decimale:

\includegraphics[scale=0.6]{decoder}

Ad esempio se in input c'è il numero 4, cioè 100 verrà attivata l'uscita 4.

\subsection{Multiplexer}

\includegraphics[scale=0.6]{multiplexer}

Un multiplexer seleziona quale degli n ingressi viene mandato all'uscita attraverso uno o più ingressi di controllo $s$

Ad esempio se i bit di controllo formano $11$, verrà mandato in output l'ingresso 3.

\subsection{Demultiplexer}

\includegraphics[scale=0.6]{demultiplexer}

\subsection{ALU ad 1 bit}

\includegraphics[scale=0.4]{alu_ad_1_bit}

Un multiplexer attiva il circuito corrispondente all'operazione da eseguire, l'ingresso carryin permette di eseguire somme in complemento a due(sottrazioni) e dà in uscita un carryout in caso di riporto.

Il risultato viene fatto uscire nell'output result.

Per formare ALU a più bit, concateniamo questi circuiti.

Per eseguire le somme dobbiamo invertire B per renderlo negativo:

\includegraphics[scale=0.4]{alu_binvert}

Usando la sottrazione possiamo anche implementare l'operazione $<$ in quanto se faccio $(a-b)$, verifico il bit più significativo, se è 1 vuol dire che la differenza è negativa e quindi $a < b$


\end{document}