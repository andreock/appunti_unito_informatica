\documentclass[a4paper]{article}
\usepackage{geometry}

\usepackage{graphicx}
\graphicspath{ {./images/} }
\usepackage[table]{xcolor}
\setlength{\parindent}{10pt}
\usepackage{amsfonts}
\usepackage{amsmath,amssymb}
% Choose a conveniently small page size
\usepackage{setspace}
% Using \doublespacing in the preamble 
% changes text to double line spacing
\doublespacing
\title{Stringhe e relazioni}
\author{Andrea Canale}

\begin{document}
\maketitle
\tableofcontents
\section{Sequenze e parole}
Dato un alfabeto di simboli, possiamo formare una parola cioè una sequenza di simboli.

Ad esempio: $A=\{la, sul, tetto, gatta\}$ formerà la parola "la gatta sul tetto".

L'insieme di tutte le parole che si possono formare sull'alfabeto A viene indicato come $A^*$.

Questo insieme è infinito perchè possiamo sempre aggiungere un simbolo ad un parola.

\section{Numero di Gödel}

Per un alfabeto generico A, finito o numerabile:

\begin{itemize}
	\item Numeriamo i simboli $ka$ $\forall a \in A$
	\item Ogni parola $a_1, ..., a_n$  diventa il numero: $2^{ka_1} \cdot 3^{ka_2} \cdot 5^{ka_3} \cdot ... \cdot p_n^{ka_n}$
\end{itemize}

Dove $p_n$ è un numero primo.

Il numero che otteniamo da questa moltiplicazione è detto numero di Gödel

\section{Operazioni su parole}

L'insieme delle parole $A^*$ ha due operazioni:

\begin{itemize}
	\item Inversione: $w \rightarrow W^r$, ad esempio $\text{felice}^r=\text{ecilef}$
	\item Concatenazione: $w_1, w_2 \rightarrow w_1w_2$, ad esempio: $w_1=\text{abra}$ $w_2=\text{cadabra}$ $w_1w_2=\text{abracadabra}$
\end{itemize}

Notiamo che l'insieme $A^*$ è un \textbf{monoide} cioè una struttura algebrica simile ad un gruppo che non ha un inverso.

Questo perchè, per la concatenazione vale l'associatività e ha come elemento neutro $\lambda$ cioè la parola vuota che non cambia il risultato della concatenazione.

Inoltre, per il monoide $A^*$ esiste la funzione lunghezza che ritorna la lunghezza di una parola. Questa funzione è un \textbf{omomorfismo}.

\subsection{Monoidi}

Un monoide $M$ è una struttura algebrica che rispetta 3 proprietà:

\begin{itemize}
	\item $ a * b \in M $
	\item Vale l'associatività
	\item Esiste un elemento neutro
\end{itemize}

Un gruppo è un monoide dotato d'inverso

\section{Sequenze}

Una sequenza è una funzione dove il dominio è sottoinsieme degli interi.

Viene indicata come $S(n)$ dove $n$ è il numero massimo nel dominio

La sua cardinalità è $n!$.

Le stringhe sono sequenze formate da un alfabeto di partenza.
\section{Classificazione di sequenza}

\subsection{Sequenze crescenti}

Una sequenza è crescente se: $s(n) < s(n+1) < s(n+2)$

\subsection{Sequenza decrescenti}

Una sequenza è decrescente se: $s(n) > s(n+1) > s(n+2)$

\subsection{Sequenza non crescenti}

Una sequenza è non crescente se: $i < j$ e $S_i \geq S_j$

\subsection{Sequenza non decrescenti}

Una sequenza è non decrescente se: $i < j$ e $S_i \leq S_j$

\section{Sottosequenze}

Data una sequenza, una sottosequenza è un sottoinsieme di quella sequenza.

Ad esempio:

Sequenza $A=\{a,a,b,c,q\}$

Si possono ottenere le seguenti sottosequenze: $\{\{a,b\}, \{b,c\}, \{b,q\}, ...\}$

\subsection{Sottosequenze note}

Una sequenza di soli numeri può essere scritta come: $[n] = \{1, ..., n\}$

Una famiglia d'insiemi di lunghezza n può essere scritta come: $A^n$


\section{Relazione}
Una relazione binaria tra un insieme X e un insieme Y è l'insieme dei prodotti cartesiano tra X e Y. Denotiamo l'insieme delle relazioni con $R$.


\section{Proprietà delle relazioni}

\subsection{Relazioni riflessive}

Una relazione è riflessiva se esiste $(x,x) \in R$ $\forall x \in X$ dove $R$ è la relazione $XxY$

\subsection{Relazioni transitive}

Una relazione è transitiva se vale $(x,y) \in R$ e $(y,z) \in R$, allora $(x,z) \in R$ $\forall x,y,z \in R$

\subsection{Relazioni simmetriche}

Una relazione è simmetrica se $(x,y) \in R$ e $(y,x) \in R$ $\forall x,y \in R$

\subsection{Relazioni antisimmetriche}

Una relazione è antisimmetrica se $x \neq y$ allora $(x,y) \notin R$ oppure $(y,x) \notin R$, allora y non è in relazione con x.

Notiamo che se $x=y$, la relazione si considera antisimmetrica.

Ad esempio:

$S=\{(0,0), (1,1), (2,2), (3,3)\}$ è antisimmetrica perchè $\forall x,y \in S$, $x=y$

\section{Relazioni d'ordine/totali}

Una relazione è definita d'ordine se valgono tre proprietà:

\begin{itemize}
	\item Transitiva
	\item Riflessiva
	\item Antisimmetrica
\end{itemize}

Esempi di relazioni d'ordine:

\begin{itemize}
	\item P(x) rispetto all'inclusione
	\item $\leq$ su $\mathbb{N}$
	\item Su $\mathbb{Z}$, $x|y$
\end{itemize}

\section{Chiusura transitiva}

Data una relazione binaria $R$, definitiamo la chiusura transitiva $R^{'} = T(r)$ come la più piccola relazione transitiva e riflessiva che contiene R

$R^{'} = \bigcap \{S|S \text{ è riflessiva e transitiva }, R \subset S \}$

\section{Relazioni d'equivalenza}

Una relazione di equivalenza è una relazione che soddisfa 3 proprietà:

\begin{itemize}
	\item è riflessiva
	\item è transitiva
	\item è simmetrica
\end{itemize}

\section{Congruenza}

Una congruenza su X è una relazione di equivalenza che mette in relazione due insiemi.


\end{document}