\documentclass[a4paper]{article}
\usepackage{geometry}

\usepackage{graphicx}
\graphicspath{ {./images/} }
\usepackage[table]{xcolor}
\setlength{\parindent}{10pt}
\usepackage{amsfonts}
\usepackage{amsmath,amssymb}
% Choose a conveniently small page size
\usepackage{setspace}
% Using \doublespacing in the preamble 
% changes text to double line spacing
\doublespacing
\usepackage{setspace} 
\title{Equivalenza logica}
\author{Andrea Canale}

\begin{document}
	
\maketitle

\begin{spacing}{2.8}
	\tableofcontents
\end{spacing}
\section{Tautologia}

La tautologia è una formula logicamente valida, cioè è vera per ogni valutazione delle lettere proposizionali.

Ad esempio $ A \lor A \Longleftrightarrow A $

Questo perchè non esiste un caso dove $ A \lor A $ è vera e A è falsa.

Si legge $ A \lor A \Longleftrightarrow A $ come $ A \lor A $ se è solo se A

L'operatore logico che usiamo per la tautologia è $ \Longleftrightarrow $

Una tautologia si può scrivere come $ \models A \land B $

\subsection{Leggi di De Morgan}

Un esempio molto importante di tautologia sono le leggi di De Morgan:

$$ \neg(p \lor q) \equiv \neg p \land \neg q $$
$$ \neg(p \land q) \equiv \neg p \lor \neg q $$

\section{Conseguenza logica}

Una proposizione Q è una conseguenza logica di un insieme di premesse P se e solo se, ogni volta che tutte le premesse in P sono vere, anche q deve essere vera.

La differenza con la tautologia è che la tautologia è sempre vera mentre la conseguenza logica dipende dalle premesse P.

L'operatore logico della conseguenza logica è $ \rightarrow $

Una conseguenza logica si può scrivere come $ A \models B $

Esempio: Data la frase: "Se piove, la strada è bagnata", può essere divisa in 2 parti: La premessa e la conseguenza.

La premessa(P) è "Se piove"
La conseguenza(Q) è "La strada è bagnata"

Quindi possiamo renderla conseguenza logica scrivendo $ P \rightarrow Q $

\section{Derivazione}

Un argomento è una serie proposizioni che concludono una proposizione scritte come: 

\[
\begin{array}{ r l }
	& p_1 \\
	& p_2 \\
	& \vdots \\
	& p_n \\
	\cline{2-2}
	\therefore & q
\end{array}
\]
Dove $ p_1, p_2, ..., p_n $ sono gli argomenti(premesse) mentre q è la conclusione.

\textbf{Un argomento è valido se la conclusione segue le ipotesi e ciò può essere dimostrato attraverso le regole d'inferenza.}

\subsection{Regole di inferenza note}

\subsubsection{Modus ponens}
\[
\begin{array}{ r l }
	& p \rightarrow q \\
	& p \\
	\cline{2-2}
	\therefore & q
\end{array}
\]

\subsubsection{Modus tollens}
\[
\begin{array}{ r l }
	& p \rightarrow q \\
	& \neg q \\
	\cline{2-2}
	\therefore & \neg p
\end{array}
\]

\subsubsection{Addizione}
\[
\begin{array}{ r l }
	& p \\
	\cline{2-2}
	\therefore & p \lor q
\end{array}
\]\\

\subsubsection{Semplificazione}
\[
\begin{array}{ r l }
	& p \land q \\
	\cline{2-2}
	\therefore & p
\end{array}
\]

\subsubsection{Congiunzione}
\[
\begin{array}{ r l }
	& p \\
	& q \\
	\cline{2-2}
	\therefore & p \land q
\end{array}
\]

\subsubsection{Silogismo ipotetico}
\[
\begin{array}{ r l }
	& p \rightarrow q \\
	& q \rightarrow r \\
	\cline{2-2}
	\therefore & p \rightarrow r
\end{array}
\]

\subsubsection{Silogismo disgiuntivo}
\[
\begin{array}{ r l }
	& p \lor q \\
	& \neg p \\
	\cline{2-2}
	\therefore & q
\end{array}
\]

\section{Insieme universo}

C'è un insieme universale $U$(universo) che contiene tutti gli elementi e tutti gli insiemi esistenti. Si assume che ogni insieme possa contenere solo elementi che appartengono anche ad $U$.

Questo ci porta al \textbf{paradosso di Russell} che denota i limiti della logica classica.

\subsection{Paradosso di Russell}

L'insieme di tutti gli insiemi che non appartengono a sé stessi appartiene a sé stesso se e solo se non appartiene a sé stesso. Questo perchè un insieme è sempre sottoinsieme di se stesso tuttavia se noi imponiamo che non sia così, è impossibile decidere se $R \in R$.

Definiamo l'insieme $R = def\{X | X \notin X\}$

Abbiamo due casi:
\begin{itemize}
	\item $R \in R$, allora vuol dire che $R \notin R$ perchè abbiamo la condizione $x \notin x$
	\item $R \in R$, allora vuol dire $R \in R$
\end{itemize}

Questa è una contraddizione ed è chiamato paradosso di Russell.

%Può essere anche espressa tramite \textbf{il paradosso del barbiere}: «In un villaggio vi è un solo barbiere, un uomo ben sbarbato, che rade tutti e solo gli uomini del villaggio che non si radono da soli. La domanda è: il barbiere si fa la barba da solo?»

\section{Quantificatori universali}

\subsection{Quantificatore "Per ogni"}

Il quantificatore "per ogni" $ \forall $, indica che una proposizione è vera per ogni valore di un insieme

\subsubsection{Regole d'inferenza per il qualificatore $ \forall $}

Usando la regola d'eliminazione abbiamo:
\[
\begin{array}{ r l }
	& \forall xP(x) \\
	\cline{2-2}
	\therefore & P(u) \text{per ogni u $ \in $ U}
\end{array}
\]

Usando la regola d'introduzione abbiamo:

\[
\begin{array}{ r l }
	& P(u) \text{per ogni u $ \in $ U} \\
	\cline{2-2}
	\therefore & \forall xP(x)
\end{array}
\]

Dove u è un elemento generico indistinguibile dagli altri dell'insieme universo(e che può essere scambiato con $ x $)

\subsection{Quantificatore esiste}

Il quantificatore "esiste" $ \exists $, indica che per almeno un elemento di un insieme, la proposizione è vera

\subsubsection{Controesempio}

Se troviamo un valore del dominio di discorso che rende falso il quantificatore esistenziale, possiamo concludere che quell'elemento è un controesempio.

\subsubsection{Regole d'inferenza per il qualificatore $ \exists $ }

Usando la regola d'eliminazione abbiamo:
\[
\begin{array}{ r l }
	& \exists xP(x) \\
	\cline{2-2}
	\therefore & P(u) \text{per qualche u $ \in $ U}
\end{array}
\]

Usando la regola d'introduzione abbiamo:

\[
\begin{array}{ r l }
	& P(u) \text{per qualche u $ \in $ U} \\
	\cline{2-2}
	\therefore & \exists xP(x)
\end{array}
\]

\section{Leggi di De Morgan generalizzate}

Esistono anche le leggi di De Morgan che valgono per il qualificatore esistenziale e quello universale e sono:

$$ \neg(\forall x P(x)) \equiv \exists x \neg P(x) $$
$$ \neg(\exists x P(x)) \equiv \forall x \neg P(x) $$

\section{Qualificatori innestati}

Possiamo anche innestare i qualificatori universali ed esistenziali, per formare proposizioni del tipo: $ \forall x \exists y P(x,y) $

Il dominio del discorso è univoco per entrambi i qualificatori, ad esempio: $ \mathbb{Z} $ 

\textbf{Il dominio di discorso non diventa prodotto cartesiano $\mathbb{Z} x \mathbb{Z}$}
\end{document}