\documentclass[a4paper, 10pt]{article}
\usepackage{geometry}
\usepackage{graphicx}
\graphicspath{ {./grammatiche_imgs/} }
\usepackage[table]{xcolor}
\setlength{\parindent}{10pt}
\usepackage{amsfonts}
\usepackage{amsmath,amssymb}
% Choose a conveniently small page size
\usepackage{setspace}
% Using \doublespacing in the preamble 
% changes text to double line spacing
\onehalfspacing
\usepackage{listings}
\usepackage{color}

\title{Grammatiche e linguaggi}
\author{Andrea Canale}

\begin{document}
	\maketitle
	\tableofcontents
\section{Grammatica}

Una grammatica è costituita dà:

\begin{itemize}
	\item Un insieme finito di simboli non terminali $N$
	\item Un insieme finito di simboli terminali $I$
	\item Un insieme di produttorie
	\item Un simboli di partenza $\sigma$
\end{itemize}

Denotiamo con $G=(N, T, P, \sigma)$
\\\\
Una produttoria associa un simbolo non terminale ad un simbolo terminale oppure ad una stringa formata da simboli terminali e non terminali
\\\\
Un linguaggio è l'insieme di tutte le parole che si possono formare con questa grammatica.
\\\\
Una parola è formata da una grammatica quando non abbiamo più simboli non terminali nella stringa.

Esempio:

Data la grammatica:

$$Frase \rightarrow SN\ SV $$
$$SN \rightarrow Articolo\ N$$
$$N \rightarrow Studente\ |\ Libro$$
$$Articolo \rightarrow Il\ |\ lo$$
$$SV \rightarrow Leggi\ SN\ | Brucia\ SN $$

Dove Frase è il simbolo d'inizio, 
possiamo scrivere la frase: Lo studente legge il libro.
Notiamo che sulla sinistra abbiamo i simboli non terminali e sulla destra abbiamo le produttorie, dove la pipe significa "oppure"

\section{Derivazioni}

Una stringa $s_1$ è derivabile da un altra $s_2$ se nella loro grammatica esiste una produttoria(o un insieme di produttorie) tale che si riesce ad ottenere la stringa $s_2$ partendo da $s_1$.
\\\\
Il linguaggio generato da una grammatica consiste in tutte le stringhe derivate in maniera finita. \textbf{Una parola derivata infinite volte non ne fa parte.}
\\\\
Si dice che una parola è derivata da una grammatica se si può scrivere come attraverso una grammatica.
\section{Sintassi BNF}

La sintassi BNF(o backus normal form oppure Backus-Naur form) è una sintassi per produrre grammatiche comoda per essere trascritta su computer.

La produttoria $S \rightarrow T$ si scrive come $S ::= T_1 | T_2 | ... | T_n$
 
\section{Tipi di grammatiche}

\subsection{Grammatiche di tipo 0}

Grammatiche molto generali, si basana su insieme ricorsivamente numerali.

\subsection{Grammatiche di tipo 1}

$A \sigma B \rightarrow A\sigma$

Grammatiche contestuali(context-sensitive), c'è almeno una termine nonterminale in una stringa.

Inoltre la parola vuota non può essere usata in una produttoria.

\subsection{Grammatiche di tipo 2}

$A \rightarrow \sigma$

Grammatiche context-free. C'è sempre un simbolo non terminale sulla sinistra. A destra c'è qualsiasi simbolo.

\subsection{Grammatiche di tipo 3}

Grammatiche regolari. Formati da:

$A \rightarrow \sigma$ oppure $A \rightarrow aB$ oppure $A \rightarrow \lambda$

Sono equivalenti agli automi a stati finiti. 
\\\\
\textbf{$A$ e $B$ in questi esempi sono non terminali mentre $\sigma$ è un terminale.}

Un linguaggio si identifica in base al tipo di grammatica.
\\\\
Notiamo inoltre che due grammatiche sono equivalenti se generano lo stesso linguaggio.

\textbf{Notiamo che una grammatica regolare è anche context-free e che una grammatica context-free senza forme $A \rightarrow \lambda$ è una grammatiche context-sensitive. Una grammatica di questo tipo è anche regolare}

\section{Automi a stati finiti non deterministici}

Possiamo convertire una grammatica in automi a stati finiti che ha le seguenti caratteristiche:

$$
	\begin{tabular}{||c | c||} 
		\hline
		Grammatica & Automi  \\
		\hline
		Insieme dei simboli terminali & Alfabeto di input \\
		\hline
		Stati & Insieme dei simboli non terminali \\
		\hline
		Stato iniziale & Simboli di partenza \\
		\hline
		Funzione di transizione & Produttorie \\
		\hline
		Stati accettanti & Simboli S tali che $S \rightarrow \lambda $ \\ 
		\hline
	\end{tabular}
$$

L'insieme delle stringhe generate da una grammatica è lo stesso accettato da un automa di questo tipo.

Tuttavia alcune grammatiche non possono essere rappresentate con automi a stati finiti deterministici, ad esempio:

\includegraphics[width=0.7\linewidth]{grammatica_esempio_automa}

Produrebbe un automa del tipo: 

\includegraphics[width=0.4\linewidth]{automa1}

Tuttavia la produttoria $C \rightarrow b$ non può essere rappresentata. Allora introduciamo uno stato ausialiario $F$ tale che: $$C \rightarrow bF \text{ e } F \rightarrow \lambda$$

Tuttavia questo stato non ha successive transizioni e quindi il suo comportamento è "imprevedibile" per questo un automa del genere è detto non deterministico.

\includegraphics[width=0.4\linewidth]{grammatiche_imgs/automa2}

Un automa a stati finiti non deterministico è caratterizzato da: 

\begin{itemize}
	\item Un insieme finito di simboli $I$ in entrata
	\item Un iniseme finito di stati $S$
	\item Una funzione $f: S x I \rightarrow P(S)$ per lo stato successivo
	\item Una sottoinsieme di $S$ di stati accettanti $A$
	\item Uno stato iniziale $\sigma$
\end{itemize}

Notiamo che non tutti gli stati hanno una funziona di transizione associata.

Perchè una stringa sia accettata in un automa a stati finiti non deterministici dobbiamo controllare che: \begin{itemize}
	\item Esiste una sequenza di stati che determina la stringa(tra le varie possibili)
	\item Lo stato finale è uno stato accettante
\end{itemize}

Data una grammatica regolare $G$, un automa non deterministico a stati finiti appropriato, accetta l'intero linguaggio generato da $G$.

\textbf{Un automa di questo tipo accetta solo linguaggi generati da grammatiche regolari}

Un linguaggio è regolare se e solo se esiste un automa non deterministico che accetta tutte le stringhe di quel linguaggio.

L'intersezione tra linguaggi regolare è regolare. Inoltre l'automa prodotto da questa intersezione è l'automa definito dall'unione dell'input $I_1, I_2$

Possono esistere più di $2^n$ stati.

\section{Da automi non deterministici a automi deterministici}

Per convertire un automa a stati finiti non deterministici ad un automa a stati finiti deterministici seguiamo la seguente tabella:


$$
\begin{tabular}{||c | c||} 
	\hline
	Non deterministico & Deterministico  \\ [0.5ex] 
	\hline
	Simboli in input & Simboli in input \\
	\hline
	Stati & Insieme potenza di tutti gli stati \\
	\hline
	Stato iniziale & Stato iniziale \\
	\hline
	Funzione di transizione & $\bigcup_{S \in X} f(S,x) = Y$ \\
	\hline
	Stati accettanti & Insieme potenza degli stati accettanti \\ [1ex] 
	\hline
\end{tabular}
$$

\end{document}