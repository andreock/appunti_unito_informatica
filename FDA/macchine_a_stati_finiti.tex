\documentclass[a4paper, 10pt]{article}
\usepackage{geometry}
\usepackage{graphicx}
\graphicspath{ {./macchine_a_stati_finiti_imgs/} }
\usepackage[table]{xcolor}
\setlength{\parindent}{10pt}
\usepackage{amsfonts}
\usepackage{amsmath,amssymb}
% Choose a conveniently small page size
\usepackage{setspace}
% Using \doublespacing in the preamble 
% changes text to double line spacing
\onehalfspacing
\usepackage{listings}
\usepackage{color}

\title{Macchine a stati finiti}
\author{Andrea Canale}

\begin{document}
	\maketitle
	\tableofcontents
	
\section{Macchine a stati finiti}

Una macchina a stati finiti è un modello di macchina con memoria che ha i seguenti requisiti:

\begin{itemize}
	\item Un insieme finito di simboli $I$ in entrata
	\item Un iniseme finito di simboli $O$ in uscita
	\item Un insieme finito di stati $S$
	\item Una funzione $f: S x I \rightarrow S$ per lo stato successivo
	\item Una funzione $g: S x I \rightarrow O$ per l'output
	\item Uno stato iniziale $\sigma$
\end{itemize}

$M=(I, O, S, f, g, \sigma)$

Queste macchina seguono la funzione $f$ applicata alla stringa in input, cambiando lo stato della macchina, e poi la funzione $g$ in base all'ultimo stato dà la stringa in output

\includegraphics{flip_flop}

Notiamo che la freccia $\rightarrow$ indica lo stato iniziale. Abbiamo poi i due stati e tutte le sequenze in input che l'automa accetta. Alcune sequenze notiamo che garantiscono il cambio di stato.

Le sequenze sono della forma bit in ingresso/bit in uscita

\section{Automi a stati finiti}

Un automa a stati finiti riconosce insiemi di parole su un alfabeto scelto precedentemente.

Un automa a stati finito è una macchina a stati finiti che ha come insieme d'uscita $\{0, 1\}$ e l'output viene stabilito in base all'ultimo stato. Gli stati che danno output 1 sono detti di accettazione.

\includegraphics{automi_a_stati_finiti}

Gli stati d'accettazione sono cerchiati due volte. In questo caso non si scrive l'output sopra la freccia perchè si capisce dallo stato se è d'accettazione oppure no.

Notiamo che può succedere che una sequenza non sia adatta per un automa. In questo caso l'automa non accetta quella sequenza. Questo accade se la sequenza e l'automa non lavorano sullo stesso alfabeto ad esempio.

Inoltre una stringa si definisce accettata dall'automa se l'ultimo stato in cui termina è uno stato accettante.

\textbf{Due automi $A$ e $A^{'}$ sono equivalenti se l'insieme di parole che accettano sono uguali $Ac(A) = Ac(A^{'})$}




\end{document}