\documentclass[a4paper]{article}
\usepackage{geometry}

\usepackage{graphicx}
\graphicspath{ {./images/} }
\usepackage[table]{xcolor}
\setlength{\parindent}{10pt}
\usepackage{amsfonts}
\usepackage{amsmath,amssymb}
% Choose a conveniently small page size
\usepackage{setspace}
% Using \doublespacing in the preamble 
% changes text to double line spacing
\doublespacing
\usepackage[overload]{empheq} % for Solution 2
% Choose a conveniently small page size
% Load the setspace package
% Using \doublespacing in the preamble
% changes text to double line spacing
\usepackage{setspace}
\usepackage{listings}
\usepackage{color}

\definecolor{dkgreen}{rgb}{0,0.6,0}
\definecolor{gray}{rgb}{0.5,0.5,0.5}
\definecolor{mauve}{rgb}{0.58,0,0.82}

\lstset{frame=tb,
	language=C,
	aboveskip=3mm,
	belowskip=3mm,
	showstringspaces=false,
	columns=flexible,
	basicstyle={\small\ttfamily},
	numbers=none,
	numberstyle=\tiny\color{gray},
	keywordstyle=\color{blue},
	commentstyle=\color{dkgreen},
	stringstyle=\color{mauve},
	breaklines=true,
	breakatwhitespace=true,
	tabsize=3
}


\title{Tecniche di dimostrazione}
\author{Andrea Canale}

\begin{document}
\maketitle
\tableofcontents

\section{Dimostrazione diretta}
La dimostrazione diretta si avvale di calcoli o fatti che dimostrino una determinata proprietà.

Quando usiamo le regole di inferenza(ad esempio per dimostrare una conseguenza logica), può essere rappresentata dal seguente schema: 

\[
\begin{array}{ r l }
	& p_1 \\
	& p_2 \\
	& \vdots \\
	& p_n \\
	\cline{2-2}
	\therefore & q
\end{array}
\]

Dove $p_1, p_2, ...$ sono le assunzione e q la conclusione logica.

\section{Dimostrazione per assurdo}

La dimostrazione per assurdo cerca di trovare un caso dove la proposizione sia falsa in modo da dimostrare che una conseguenza logica non sia vera.

La dimostrazione per assurdo cerca di trovare una conclusione della forma $ C \land \neg C$

\section{Dimostrazione per contrapposizione}

La dimostrazione per contrapposizione serve a verificare proposizioni usando la forma 

$\neg B \rightarrow \neg A$

\section{Dimostrazione per casi}

La dimostrazione per casi viene usata quando la proposizione ammette diversi casi che devono essere tutti veri perchè la proposizione sia vera.

$$(A_1 \rightarrow B) \land ... \land (A_n \rightarrow B)$$

Ogni caso può essere dimostrato con una tecnica di dimostrazione a scelta.

\section{Dimostrazione per induzione}

La tecnica dell'induzione serve per dimostrare proprietà che valgono per tutti i numeri naturali.

Viene usata quando la proposizione ha una forma del tipo:

\begin{alignat*}{4}[left = \empheqlbrace]
	f(0) & =n \\
	f(n) & = n \cdot f(n-1)
\end{alignat*}

Quindi abbiamo due casi: \begin{itemize}
	\item Il caso base $f(0)$ dove 0 è il valore iniziale(può essere anche $\neq 0$)
	\item Il caso generale che si riconduce ad un certo punto al caso base
\end{itemize}

Il principio d'induzione si compone in 3 passi:

\begin{itemize}
	\item Dimostrazione del caso base, dove per dimostrazione diretta dimostriamo il caso base
	\item Passo induttivo, quando assumiamo che $f(n)$ sia vera
	\item Dimostrazione del caso generale, dimostrando per dimostrazione diretta $f(n+1)$ usando l'assunzione del passo induttivo
\end{itemize}

\subsection{Invariante}

L'invariante di un ciclo iterativo sfrutta la proprietà dell'induzione sul numero di volte che un ciclo viene eseguito per dimostrare che dopo $k$ iterazioni vale una certa proprietà.

Ad esempio:

\begin{lstlisting}
	int x = 0;
	int y = 0;
	while(x < n){
		x = x + 1;
		y = x * y;
	}
\end{lstlisting}

Questo codice è invariante e rappresenta $y=x!$

Per dimostrarlo usiamo l'induzione sul numero di volte cheA un ciclo viene eseguito, indicato come $k$.

Abbiamo 3 passi fondamentali:

\begin{itemize}
	\item Caso base, dimostriamo che per k = 0, la proprietà vale. 
	\item Passo induttivo, assumiamo che p(k) vale
	\item Caso generale, dimostriamo che p(k+1) vale 
\end{itemize}

Questa proprietà dell'invariante è utile per controllare la correttezza del codice.

\subsection{Induzione forte}

L'induzione forte viene usata quando vogliamo dimostrare che tutti i casi precedenti al passo induttivo sono validi e non solo al caso base.

Se vogliamo dimostrare una proprietà $S(n)$ dove il dominio di discorso sono i numeri naturali maggiori o uguali a $n_0$, possiamo

\begin{itemize}
	\item Dimostrare $S(n_0)$
	\item Nel passo induttivo assumiamo che $S(k)$ sia vera dove $n_0 \leq k < n$
	\item Dimostriamo che $S(n)$ è vera dove $n > n_0$
\end{itemize}

Concludiamo che $S(n)$ è vera per ogni $n > n_0$

\subsection{Principio del minimo/Principio del buon ordinamento}

Il principio di buon ordinamento impone che in ogni insieme non vuoto di numeri naturali esiste sempre un elemento minimo.

Da ciò ne ricaviamo che se la proprietà P è vera per qualche numero naturale, allora c’è un minimo numero naturale n tale che P(n).

\textbf{L'induzione forte e il principio del minimo sono equivalenti al principio d'induzione}
\end{document}